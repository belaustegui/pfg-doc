El presente proyecto tiene como objetivo la creación de la infraestructura de
soporte  para un portal de datos abiertos.  Concretamente
el portal será el nuevo Land Portal, propiedad del IFAD (Fondo Internacional para
el Desarrollo Agrícola) y perteneciente a la ONU (Organización de las Naciones
Unidas).  Este Proyecto Fin de Grado forma parte del proyecto \textit{Rebuilding
IFAD's LandPortal - RFQ/2013/016/SC} desarrollado por el grupo de investigación
WESO (Web Semantics Oviedo) y la empresa SB Consulting.

El sistema resultante está formado por dos secciones principales:
\begin{itemize}
	\item
		Por un lado una sección orientada a los datos (LandBook), que permite
		la publicación y consulta de catálogos de datos en varios formatos
		como JSON, XML o RDF.  Los catálogos de datos almacenados en el
		sistema son utilizados, entre otros, por unas visualizaciones que los
		presentan de forma gráfica y atractiva con el objetivo de facilitar su
		interpretación por parte de los usuarios.
	\item
		Por otro lado una sección social (LandDebate) que permite la creación de
		noticias, eventos y debates con los que fomentar la participación e
		implicación de los usuarios.  Esta sección también alberga el blog
		del nuevo Land Portal.
	
\end{itemize}

Una parte importante del sistema es su soporte a la internacionalización en todas 
sus secciones.  La internacionalización es completa e incluye tanto las interfaces
de usuarios como el propio modelo de datos.  Esta característica permite que el
nuevo Land Portal sea utilizado por multitud de personas y organizaciones
pertenecientes a diversos países.

El sistema está desarrollado utilizando varias tecnologías diferentes: varios lenguajes
de programacion (PHP, Python JavaScript), un gestor de contenidos o CMS (Drupal),
un catálogo de datos (CKAN), una base de datos relacional (MySQL), una base de
datos no relacional (Virtuoso) y un framework de desarrollo web (Flask).

Para la realización de este proyecto se han utilizado los conocimientos adquiridos
durante los estudios del Grado en Ingeniería Informática del Software.