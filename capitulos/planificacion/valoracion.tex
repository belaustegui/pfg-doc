Como se ha podido comprobar existen varias diferencias entre la planificación inicial (sección \ref{planificacion:inicial}) y la planificación real (sección \ref{planificacion:real}).  A pesar de que la planificación inicial nunca es perfecta y es normal que la planificación real no se ajuste completamente, en ésta sección se realizará una valoración de ambas explicando las razones para dichas divergencias.

\begin{itemize}
	\item
		Durante el desarrollo del proyecto ha habido algunos cambios en los requisitos iniciales, lo que ha obligado a reestructurar algunos componentes del sistema que ya estaban implementados o a reorganizar futuras tareas.
	\item
		Como se puede comprobar en la planificación real, se han añadido tres hitos para demostraciones por parte del cliente.  El cliente del proyecto ha necesitado encontrar financiación para seguir adelante con futuros proyectos, uno de los mecanismos para conseguir inversores ha sido realizar varias demostraciones del portal que se ha desarrollado en éste proyecto.  La necesidad de realizar éstas demostraciones ha obligado a reestructurar algunas tareas para finalizar partes del sistema y que se pudieran ir enseñando.
	\item
		El Punto de Entrada de Datos ha tenido varios problemas de rendimiento a lo largo de su desarrollo.  La realización de pruebas de rendimiento y su posterior solución ha alargado el desarrollo de éste componente, lo que ha repercutido en el resto de componentes del sistema.  El resultado de éstas pruebas puede verse con detalle en la sección ``\nameref{pruebas:rendimiento}'' perteneciente al capítulo \ref{chapter:desarrollo_pruebas}.
	\item
		El desarrollo como parte de un equipo hace que unas partes del sistema sean dependientes de otras que, en muchas ocasiones, son realizadas por otras personas.  La coordinación con el resto de miembros del equipo ha producido variaciones en las fechas de inicio y fin de algunos componentes e hitos.
\end{itemize}