En éste capítulo se analizarán las posibles alternativas a los distintos componentes del sistema y posteriormente se seleccionarán las que se utilizarán para el desarrollo del sistema final.

\section{Alternativas evaluadas}

\subsection{Gestor de contenidos}
Un gestor de contenidos (\textit{CMS}) permite publicar, editar y modificar de forma sencilla los contenidos de una página web.  La mayoría de gestores de contenidos también se encargan de la gestión de los usuarios, los roles de usuario y permisos de cada uno de ellos.

Hay multitud de gestores de contenido disponibles en el mercado. En esta sección se van a analizarán varios de ellos y posteriormente se explicará cual se ha escogido para su utilización en éste proyecto.


\subsubsection{Joomla}
Joomla\footnote{\url{http://www.joomla.org}/} es un gestor de contenido de código abierto, con licencia GPL y escrito en PHP.  Fue creado en 2005 como un fork de otro gestor de contenido llamado \textit{Mambo}\footnote{\url{http://www.mamboserver.com/}}.\\
Joomla es utilizado en sitios web de gran relevancia, como la página web de la Universidad de Harvard\footnote{\url{http://gsas.harvard.edu/}}.

\begin{figure}[h]
\centering
\includegraphics[height=2.5cm]{logos/joomla_logo}
\caption{Logotipo de Joomla}
\end{figure}

Las principales ventajas de Joomla respecto a otros gestores de contenido son su sencillez y su capacidad de extensión.\\
Joomla está diseñado utilizando técnicas de programación orientada a objetos y aplicando varios patrones de diseño de software, lo que hace que su código esté relativamente bien formado.\\
Por otra parte Joomla soporta cinco tipos diferentes de extensiones (componentes, plugins, plantillas, módulos e idiomas).  Cada uno de estos tipos de extensión tiene un comportamiento y una finalidad diferente, lo que permite que sea posible adaptar el funcionamiento del gestor de contenidos a cada necesidad particular.


\subsubsection{Wordpress}
Wordpress\footnote{\url{https://wordpress.org/}} es un gestor de contenido de código abierto, con licencia GLPv2 y escrito en PHP.  Fue creado en mayo de 2003 como un fork de otro gestor de contenido llamado \textit{b2/cafelog}.\\
En la actualidad Wordpress se utiliza en más de 68 millones de sitios web, entre ellos el sitio web del New York Times\footnote{\url{http://www.nytimes.com/}}, el sitio web de CNN\footnote{\url{http://edition.cnn.com/}} o el sitio web de Forbes\footnote{\url{http://www.forbes.com/}}.

\begin{figure}[h]
\centering
\includegraphics[height=2.5cm]{logos/wordpress_logo}
\caption{Logotipo de Wordpress}
\end{figure}

La principal ventaja de Wordpress respecto a otros gestores de contendo es su simplicidad, puesto que está orientado principalmente a la construcción de sitios orientados al blogging o a las noticias.


\subsubsection{Drupal}
Drupal\footnote{\url{https://drupal.org/}} es un gestor de contenido de código abierto creado en el año 2001.  Al igual que Joomla y Wordpress está escrito en PHP y cuenta con licencia GPLv2.\\
Como se ha mencionado anteriormente en la sección ``\nameref{estudio_situacion_actual}'' perteneciente al capítulo \ref{chapter01}, Drupal se utiliza en el portal de datos del Gobierno Británico\footnote{\url{http://data.gov.uk/}} y en el portal de datos del Gobierno de Estados Unidos\footnote{\url{http://www.data.gov/}}.

\begin{figure}[h]
\centering
\includegraphics[height=2.5cm]{logos/drupal_logo}
\caption{Logotipo de Drupal}
\end{figure}

La principal ventaja de Drupal es su flexibilidad para extender y modificar su funcionamiento.  En la actualidad cuenta con más de 15000 módulos disponibles.  Su sistema de \textit{hooks} permite crear módulos que responden a eventos llamados y ejecutados de forma automática por el \textit{core} de Drupal.





\subsection{Catálogo de datos}
Un catálogo de datos permite almacenar y organizar bajo una estructura común catálogos de datos procedentes de diversas fuentes y que pueden encontrarse en distintos formatos.


\subsubsection{CKAN}
CKAN\footnote{\url{http://ckan.org/}} (\textit{Comprehensive Knowledge Archive Network}) es una plataforma de código abierto para la construcción de portales de datos y creada por la OKFN (\textit{Open Knowledge Foundation})\footnote{\url{https://okfn.org/}} que permite publicar, buscar y organizar catálogos de datos. CKAN está escrito en Python\footnote{\url{https://www.python.org/}} y utiliza PostgreSQL\footnote{\url{http://www.postgresql.org/}} como base de datos.

El funcionamiento de CKAN consiste en almacenar los catálogos de datos junto con diversos metadatos, que posteriormente son accesibles y modificables desde una interfaz web amigable para los usuarios.  Además de la interfaz web CKAN también ofrece una API que permite interactuar con otras aplicaciones y servicios de terceros.

Como se ha mencionado anteriormente en la sección ``\nameref{estudio_situacion_actual}'' perteneciente al capítulo \ref{chapter01}, CKAN se utiliza en varios portales de datos de gran envergadura, como pueden ser el portal de datos del Gobierno Británico o el portal de datos del Gobierno de Estados Unidos.


\subsubsection{DKAN}
DKAN\footnote{\url{http://nucivic.com/dkan/}} es una plataforma basada en CKAN y Drupal para facilitar la publicación de datos.  A diferencia de CKAN, DKAN está escrito utilizando el lenguaje de programación PHP.

La principal ventaja de DKAN es la estrecha integración entre el catálogo de datos (CKAN) y el gestor de contenidos (Drupal).  Ésta integración entre los dos componentes permite aprovechar las mejores características de cada uno de ellos.  Por otra parte, ésta integración también permite desplegar el catálogo de datos de una forma simple sobre una instalación de Drupal ya existente.

Un punto en contra de DKAN es su falta de madurez.  DKAN fue creado en 2012 y recientemente ha alcanzado la versión 1.0.  A pesar de todo, ha sido utilizado en algunos proyectos como el portal de datos de la Ciudad de Colonia y el portal de datos del Gobierno de Puerto Rico.


\subsubsection{Herramienta creada especialmente para la ocasión}
La funcionalidad de un catálogo de datos podría ser implementada por una herramienta creada especialmente para éste proyecto e implementada como un módulo del gestor de contenidos.

La principal ventaja de esta aproximación es que, dado que la herramienta se crearía especialmente para éste proyecto, cumpliría totalmente con las necesidades del mismo.  La principal desventaja radica en el esfuerzo requerido para implementar una herramienta de tal calibre, esfuerzo que ya viene solucionado por parte de otras herramientas existentes y de probada estabilidad.



\subsection{Servidor semántico}
Puesto que éste proyecto consiste en la creación de un portal de datos enlazados abiertos, es necesario incluir un componente que se encargue de aportar la parte de datos enlazados.  Éstos componentes suelen llamarse servidores semánticos o \textit{triple-stores}, puesto que almacenan los datos en formato RDF\footnote{\url{http://www.w3.org/RDF/}}, que modela los datos en forma de tripletas.


\subsubsection{Virtuoso Universal Server / Virtuoso OpenLink}
Virtuoso Universal Server\footnote{\url{http://virtuoso.openlinksw.com/}} es un motor de base de datos con una arquitectura híbrida que le permite ofrecer diferentes funcionalidades, que tradicionalmente han sido realizadas por diferentes productos, en un mismo componente.\\
Virtuoso fue creado en 1998 de la unión del middleware de acceso a datos \textit{OpenLink}  y el sistema de gestión de bases de datos relacionales \textit{Kubl}.

Además de la capacidad para ofrecer diferentes servicios en un mismo componente, otra gran ventaja de Virtuoso es su estabilidad y rendimiento.  
Las principales ventajas de Virtuoso son su capacidad para ofrecer diferentes servicios desde un mismo componente y su estabilidad y rendimiento. Además Virtuoso porporciona un punto de acceso SPARQL\footnote{SPARQL es un lenguaje para realizar consultas en grafos RDF de forma similar a cómo SQL sirve para realizar consultas en bases de datos relacionales.} para consultar los datos que almacena.\\
Como se indica en \cite[]{largetriplestores}, la versión 6.1 de Virtuoso ha llegado a servir 15.4 billones de tripletas simultáneamente (incluyendo el catálogo completo del portal de datos del Gobierno de Estados Unidos).

Virtuoso un producto propietario, aunque tiene una versión libre con licencia GLPLv2 que recibe el nombre de Virtuoso OpenLink\footnote{\url{http://virtuoso.openlinksw.com/dataspace/doc/dav/wiki/Main/}}. Virtuoso y Virtuoso OpenLink es utilizado activamente en varios portales de datos, entre ellos el portal de datos de la Web Foundation\footnote{El portal de datos de la Web Foundation ha sido también desarrollado por el grupo de investigación WESO y está disponible en \url{http://data.webfoundation.org/}} y la DBPedia\footnote{\url{http://wiki.dbpedia.org/}}.


\subsubsection{Stardog}
Stardog\footnote{\url{http://www.stardog.com/}} es un sistema de base de datos RDF escrito en el lenguaje de programación Java. Stardog es un producto comercial, aunque cuenta con una versión gratuita con características limitadas.

La principal ventaja de Stardog frente al resto de servidores semánticos es su capacidad para realizar inferencias sobre los datos que almacena, así como su soporte a la especificación OWL2\footnote{OWL2 es un estándar publicado en 2008 por el W3C con el objetivo de construir un modelo de marcas basado en RDF y codificado en XML.}.

Como se menciona en \cite[]{largetriplestores}, la versión 2.1 de Stardog soporta hasta 50 billones de tripletas almacenadas simultáneamente.


\subsubsection{4store}
4store\footnote{\url{http://4store.org/}} es un sistema de gestión de bases de datos y un motor de consultas que almacena datos en formato RDF. 4store está escrito en ANSI C99 y cuenta con licencia GPLv3.

Las principales ventajas de 4store son su rendimiento, su escalabilidad y su estabilidad. 4store lleva siendo usado en producción en Garlik\footnote{\url{http://www.garlik.com/}} durante 3 años. Como se puede ver en \cite[]{largetriplestores}, 4store soporta hasta 15 billones de tripletas almacenadas simultáneamente.

Por otra parte, una posible desventaja de 4store es el limitado número de características que ofrece. 4store únicamente proporciona almacenamiento RDF y consultas SPARQL.



\subsection{Visualizador de datos enlazados}
Un visualizador de datos enlazados accede a los datos que se almacenan en el servidor semántico en formato RDF para mostrarlos de una forma que atractiva para el usuario y permitir navegar a través de ellos sin necesidad de realizar complejas consultas SPARQL.


\subsubsection{Pubby}
Pubby\footnote{\url{http://wifo5-03.informatik.uni-mannheim.de/pubby/}} es un visualizador de datos enlazados escrito en el lenguaje Java.  Pubby fue creado por Richard Cyganiak y cuenta con una licencia Apache v2.

La principal ventaja de Pubby es su facilidad de configuración y su demostrada estabilidad. Tal y como Chris Bizer explica en \cite[]{dbpedia-architecture}, originalmente Pubby fue utilizado como \textit{frontend} de Virtuoso en DBPedia.  Posteriormente Pubby fue sustituido por consumidores especializados en HTML y RDF por cuestiones de rendimiento.

La principal desventaja de Pubby es su estado de abandono. Al momento de realizar éste proyecto, la última actualización de Pubby data de enero de 2011.

\subsubsection{Wesby}
Wesby\footnote{\url{https://github.com/weso/wesby}} es un visualizador de datos enlazados escrito en el lenguaje Scala. Es un desarrollo propio de WESO\footnote{\url{http://www.weso.es/}} que tiene por objetivo establecerse como alternativa que mejore lo ofrecido por Pubby.

\begin{figure}[h]
\centering
\includegraphics[width=\textwidth]{logos/wesby_logo}
\caption{Logotipo de Wesby}
\end{figure}

La principal ventaja de Wesby sobre Pubby es su apariencia y su rendimiento. Wesby ofrece una interfaz limpia y adaptable a diferentes dispositivos. Además, una característica ofrecida por Wesby es el soporte para la creación de visualizaciones para diferentes tipos de nodos dentro del grafo RDF. Otra ventaja de Wesby es que éste construye automáticamente el esquema de URIs a partir del grafo de datos presente en el servidor semántico al que se conecta.

Una posible desventaja de Wesby respecto a Pubby es su novedad. Wesby es un proyecto relativamente nuevo, y únicamente se ha utilizado en el portal de datos de la Web Foundation, por lo que su estabilidad no está totalmente probada.



\subsection{Framework de diseño web}
Un framework de diseño web facilita la tarea de diseñar la interfaz de la web del portal de forma que resulte atractiva a los usuarios.


\subsubsection{Twitter Bootstrap}
Twitter Bootstrap\footnote{\url{http://getbootstrap.com/}} es un framework de diseño web desarrollado por Mark Otto y Jacob Thornton.  Bootstrap vió la luz como proyecto de código abierto en el año 2011.

Una gran ventaja de Bootstrap respecto a otros frameworks de diseño web es el gran número de elementos prediseñados con el que cuenta. Otra característica interesante de Bootstrap es que facilita el desarrollo adaptable a diferentes dispositivos al utilizar una organización en filas y columnas de ancho variable.\\
Bootstrap es utilizado en multitud de sitios web, por lo que cuenta con una comunidad muy amplia y existe una gran cantidad de temas y documentación disponible.

Una desventaja de Bootstrap y, en general, de cualquier framework de diseño web es que obliga a realizar el desarrollo de una forma determinada para poder aprovechar sus características, desviarse del camino establecido puede provocar dificultades durante el desarrollo.


\subsubsection{ZURB Foundation}
ZURB Foundation\footnote{\url{http://foundation.zurb.com/}} es un framework de diseño web desarrollado por ZURB en 2011.

La principal ventaja de Foundation respecto a otros frameworks de diseño web es que permite diseñar la web orientada principalmente a dispositivos móviles y adaptada posteriormente a pantallas de un tamaño más grande.

Una característica de Foundation que produce al mismo tiempo ventajas y desventajas es que Foundation cuenta con menos elementos prediseñados que Bootstrap, algo que obliga al desarrollador trabajar más si quiere utilizar componentes no disponibles en el framework pero, al mismo tiempo, permite una mayor flexibilidad en los diseños.


\subsubsection{Desarrollo desde cero}
Una tercera alternativa es realizar el diseño del portal desde cero sin utilizar ningún framework.

La principal ventaja de esta solución es la total flexibilidad para desarrollar y dar la apariencia que se desee sin estar sujetos a las ataduras de un framework.

La principal desventaja es la cantidad de trabajo necesario para implementar y mantener una apariencia coherente a lo largo de todo el portal, además de el esfuerzo necesario para conseguir una interfaz adaptable a diferentes dispositivos.



\subsection{Motor de búsqueda}
Un motor de búsqueda se encarga de catalogar e indexar los contenidos de un portal, construyendo un índice inverso que permita a los usuarios encontrar la información que necesiten.


\subsubsection{Apache Solr}
Apache Solr\footnote{\url{http://lucene.apache.org/solr/}} es un motor de búsqueda desarrollado por la Apache Software Foundation\footnote{\url{http://www.apache.org/foundation/}}.  Fue creado en el año 2004 utilizando el lenguaje Java y tiene una licencia Apache v2.

Solr utiliza la librería de búsqueda Lucene\footnote{\url{http://lucene.apache.org/}}, desarrollada también por la Apache Software Foundation.  En el año 2010 los equipos de desarrollo encargados de Solr y Lucene se unieron y desde entonces es común referirse a ambos productos como \textit{Lucene/Solr}.

Un punto a favor de Solr es su madurez y su amplia comunidad de usuarios y desarrolladores, lo que permite encontrar plugins para integrarlo en multitud de productos. Solr es utilizado en multitud de proyectos de gran envergadura como Netflix\footnote{\url{https://www.netflix.com/global}}, Apple\footnote{\url{http://www.apple.com/}}, MTV\footnote{\url{http://www.mtv.com/}} o digg\footnote{\url{http://digg.com/}}.\\
Otra característica positiva de Solr es la existencia de una API de búsqueda por HTTP. La API HTTP de Solr recibe peticiones GET y permite escoger entre varios formatos de respuesta como XML o JSON.

Un punto que puede resultar negativo de Solr frente a otros motores de búsqueda es la necesidad de definir un esquema de datos.  El esquema es un fichero en formato XML que define qué estructura tendrá el índice.  A pesar de esto, es posible definir elementos dinámicos que se creen bajo demanda sin tener que figurar en el esquema.

\subsubsection{Elasticsearch}
Elasticsearch\footnote{\url{http://www.elasticsearch.org/}} es un motor de búsqueda desarrollado por la Organización Elasticsearch\footnote{\url{https://github.com/elasticsearch}}.  Fue creado en 2010 por Shay Banon utilizando el lenguaje Java.  Al igual que Solr, Elasticsearch también cuenta con una licencia Apache v2 y utiliza la librería de búsqueda Lucene.

La principales características a favor de Elasticsearch son su capacidad para construir sistemas de búsqueda distribuidos y su arquitectura \textit{schemaless}, lo que permite enviar documentos para que sean indexados sin necesidad de definir un esquema previamente.

El punto débil de Elasticsearch es su novedad. A pesar de ser utilizado en portales como Quora\footnote{\url{https://www.quora.com/}} o GitHub\footnote{\url{https://github.com/}} no cuenta con la madurez y amplia comunidad de productos más maduros como Solr.


\subsubsection{Buscador propio del gestor de contenidos}
Los tres gestores de contenidos que se han analizado anteriormente cuentan con un buscador propio integrado.  La principal ventaja de éste buscador es la completa integración con los contenidos que forman parte del CMS.  Puesto que el buscador ya forma parte del CMS, normalmente no es necesario realizar ninguna tarea de programación ni configuración para activarlo.

La gran desventaja de ésta solución es la imposibilidad de indexar y buscar todos aquellos contenidos que no formen parte del CMS,  algo que enn éste proyecto esto haría imposible buscar los contenidos pertenecientes a la sección de datos.



\section{Alternativas elegidas}
\label{chapter02:alternativas_seleccionadas}
En esta sección se explicará qué alternativas de las anteriormente descritas se han seleccionado y las razones por las que se ha llevado a cabo dicha selección.

\subsection{Gestor de contenidos}
De los tres gestores de contenidos analizados para éste proyecto, se ha seleccionado Drupal debido principalmente su madurez y flexibilidad.

A pesar de que los tres gestores de contenidos analizados se utilizan actualmente en multitud de proyectos, Drupal forma parte de varios proyectos con objetivos similares a éste, como los portales de datos del Gobierno Británico y del Gobierno de Estados Unidos ya mencionados anteriormente. Por otra parte, una característica a tener muy en cuenta es la cantidad de módulos disponibles para extender Drupal, que en éste momento supera los 15000.

Un punto importante en esta decisión es que, como Larry Garfield explica en \cite[]{pac-vs-mvc} Drupal utiliza el patrón arquitectónico PAC\footnote{Presentation Abstraction Control - \url{http://en.wikipedia.org/wiki/Presentation-abstraction-control}} para organizar su estructura y funcionamiento.  El patrón PAC es menos conocido y utilizado que el patrón MVC\footnote{Model View Controller - \url{http://martinfowler.com/eaaDev/uiArchs.html}}, lo que puede reflejarse en una mayor dificultad a la hora de realizar la implementación.


\subsection{Catálogo de datos}
Entre las tres alternativas mencionadas anteriormente, se ha seleccionado CKAN como catálogo de datos para éste proyecto.

Al igual que Drupal, CKAN se utiliza en varios portales de datos con objetivos similares al que se pretende construir en éste proyecto. CKAN permitirá ofrecer una vista completa de los conjuntos de datos incluidos en Land Portal.

La alternativa de utilizar una herramienta implementada especialmente para el proyecto se ha descartado por la gran complejidad que conlleva tanto la propia implementación como la integración con el gestor de contenidos y la creación de una interfaz de acceso a los datos.


\subsection{Servidor semántico}
Se ha seleccionado Virtuoso como servidor semántico para el nuevo Land Portal. Ésta seleccion viene motivada por varios factores que se explicarán a continuación.

Como se ha explicado anteriormente, Virtuoso ofrece un punto de acceso SPARQL, algo que se considera indispensable en cualquier portal de datos abiertos y enlazados.  Además, a diferencia de otras alternativas como Stardog, la versión libre de Virtuoso no tiene ninguna limitación en el número de conexiones que puede recibir ni en el uso de CPU, la única limitación de la versión libre es su capacidad para funcionar como una única instancia, pero ésto no debería ser un problema dada la cantidad de datos que se pretende almacenar en el sistema.

Por último, al contrario que 4store, Virtuoso ofrece un API que permite almacenar y extraer datos, lo que puede resultar de utilidad a la hora de realizar el proceso de importación y enriquecimiento de datos.


\subsection{Visualizador de datos enlazados}
Entre las dos alternativas para el visualizador de datos enlazados se ha seleccionado Wesby.

Ésta selección viene motivada principalmente por ser Wesby un desarrollo propio del grupo de investigación WESO en el que el propio autor de éste Proyecto Fin de Grado ha participado, además de por ofrecer una interfaz moderna, amigable y adaptable a diferentes dispositivos que permitirá que no desentone respecto a las demás partes del nuevo Land Portal.


\subsection{Framework de diseño web}
Como framework de diseño web se ha seleccionado Bootstrap.

La principal razón para seleccionar Bootstrap frente a Foundation es que Bootstrap cuenta con una mayor cantidad de usuarios, lo que produce una mayor comunidad y permite encontrar recursos como documentación, temas, etc. más fácilmente.

El desarrollo de la apariencia del portal sin utilizar un framework de diseño web ha sido descartado debido al gran esfuerzo que requeriría en implementar un diseño moderno, vistoso, coherente y adaptativo desde cero.


\subsection{Motor de búsqueda}
La decisión del motor de búsqueda ha sido quizás de las más complejas. La decisión final se ha decantado en favor de Apache Solr.

A pesar de que tanto Solr como Elasticsearch son utilizados en varios proyectos importantes, la mayor madurez de Solr hace más sencillo encontrar plugins e información.  En relación con la decisión del gestor de contenido, Solr cuenta con un plugin para Drupal\footnote{\url{https://drupal.org/project/apachesolr}} que lleva 7 años en desarrollo activo, por lo que su madurez y estabilidad quedan fuera de toda duda.

Debido también a la cantidad de datos con la que contará el portal tampoco parece necesario recurrir a las altas capacidades de búsqueda distribuida ofrecidas por Elasticsearch.

