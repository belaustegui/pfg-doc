En esta sección se detallarán los aspectos más importantes de la implementación del sistema, desde los lenguajes de programación y herramientas utilizadas hasta los principales problemas surgidos durante el desarrollo y sus soluciones.


\section{Lenguajes de programación utilizados}
\label{implementacion:lenguajes_programacion}

A continuación se describirán los lenguajes de programación utilizados y sus respectivas versiones.
\begin{description}
	\item[PHP]
		Para el desarrollo de los módulos pertenecientes al gestor de contenidos, así como las plantillas del tema visual, se ha utilizado el lenguaje PHP (\textit{PHP: Hypertext Preprocessor}).  La versión del lenguaje PHP utilizada es la 5.5\footnote{La versión 5.5 del lenguaje PHP incluye, entre otras características, soporte a generadores y capacidad de iteración por cualquier tipo de clave (no sólo numérica) en los bucles \textit{foreach}.  Para una mayor información al respecto se recomienda leer el anuncio oficial en \url{http://php.net/releases/5_5_0.php}}.  La única extensión de PHP utilizada ha sido APC (\textit{Alternative PHP Cache}), que permite cachear y optimizar el código intermedio de PHP con el objetivo de aumentar el rendimiento siempre que sea posible, el uso de éste caché puede verse descrito en la sección ``\nameref{actividad:framework_visualizaciones}'' perteneciente al capítulo \ref{chapter05}.
	\item[Python]
		Para el desarrollo del Punto de Entrada de Datos (o \textit{Receiver}) se ha utilizado el lenguaje Python en su versión 2.7. Se ha decidido utilizar la versión 2.7 de Python debido a que las versiones 3.x contienen varias incompatibilidades con las anteriores versiones, éstas incompatibilidades provocan que la existencia de librerías externas sea más reducida.  En el PEP (\textit{Python Enhancement Proposal}) 373\footnote{El PEP 373 está disponible para su consulta en \url{http://legacy.python.org/dev/peps/pep-0373/}} se establece que la versión 2.7 de Python estará soportada hasta el año 2020..  En la posterior sección ``\nameref{implementacion:herramientas_utilizadas}'' se explicarán las herramientas que se han utilizado para aislar el entorno y gestionar las dependencias externas.
	\item[SQL]
		Para interactuar con la base de datos se ha utilizado el lenguaje SQL, Cabe destacar que todas las consultas realizadas se adhieren a la especificación de SQL estándar y no utilizan características propias de ningún sistema de gestión de bases de datos.
	\item[Otros]
		A pesar de no ser lenguajes de programación como tal, también se incluirán algunos lenguajes que se han utilizado en varias partes del sistema.
		\begin{itemize}
			\item \textbf{XML}
				Para el intercambio de datos entre los importadores y el Punto de Entrada de Datos se ha utilizado el lenguaje de marcas XML (\textit{Extensible Markup Language}).  A pesar de la existencia de otros lenguajes con un objetivo similar (JSON es uno de ellos y se verá a continuación) se ha decidido utilizar XML para ésta tarea por su capacidad para validar la estructura del documento contra un esquema dado.
			\item \textbf{JSON}
				Para el envío de datos a las vistas y visualizaciones se ha utilizado el formato JSON (\textit{JavaScript Object Notation}).  Se ha utilizado JSON por ser un formato mucho más ligero que XML.  Además, puesto que en éstos casos los datos provienen desde dentro del sistema, se garantiza su integridad.
					\begin{enumerate}
						\item
							El formato JSON también se ha utilizado para albergar varios ficheros de configuración, como se puede ver en la ``\nameref{vista_landportal_uris}''.  A pesar de la existencia de otros formatos como YAML (\textit{YAML Ain't Markup Language}) se ha decidido utilizar JSON por consistencia con el resto de partes del sistema.
				\end{enumerate}
			\item \textbf{HTML y CSS}
				Toda la interfaz web del portal se ha desarrollado utilizando el lenguaje HTML 5 para crear el contenido de las diferentes páginas, así como el lenguaje CSS 3 para dar estilo a las mismas.
		\end{itemize}
\end{description}


\section{Herramientas utilizadas}
\label{implementacion:herramientas_utilizadas}

A continuación se describirán las herramientas utilizadas durante el desarrollo del sistema.
\begin{description}
	\item[PyCharm]
		Para el desarrollo del Punto de Entrada de Datos se ha utilizado el entorno de desarrollo integrado PyCharm en su versión 3.3 y 3.4.  PyCharm es un IDE para el desarrollo en lenguaje Python desarrollado por JetBrains\footnote{\url{http://www.jetbrains.com/pycharm/}}.
	\item[Editor de texto]
		Para el desarrollo del código PHP que forma parte de los módulos de Drupal se ha utilizado en primer lugar el editor de texto Sublime Text 2\footnote{\url{http://www.sublimetext.com/2}} y posteriormente el editor Atom\footnote{\url{https://atom.io/}}.  Se ha decidido cambiar desde Sublime Text a Atom debido a ser éste último un producto de código abierto, gratuito y con un desarrollo muy activo.
	\item[VirtualEnv]
		Se ha utilizadoo la herramienta VirtualEnv\footnote{\url{http://virtualenv.readthedocs.org/en/latest/virtualenv.html}} para crear entornos de Python aislados.  Utilizar entornos aislados permite instalar librerías y paquetes sin que colisionen entre los diferentes entornos existentes, también permite mantener múltiples intérpretes de Python funcionando de forma simultánea.
	\item[PIP]
		En conjunción con la herramienta VirtualEnv mencionada anteriormente, también se ha utilizado la herramienta PIP\footnote{\url{https://pypi.python.org/pypi/pip}} para gestionar los distintos paquetes y dependencias.  PIP permite descargar e instalar de forma automática paquetes externos junto a sus dependencias, también permite especificar todas las dependencias en un único fichero de texto.
	\item[Git y GitHub]
		Durante todo el desarrollo se ha utilizado Git\footnote{\url{http://www.git-scm.com/}} como sistema de control de versiones.  En conjunción con Git, se ha utilizado GitHub\footnote{\url{https://github.com/}} como \textit{hosting} para los distintos repositorios que forman parte del sistema.
\end{description}


\section{Problemas encontrados}
\label{implementacion:problemas_encontrados}