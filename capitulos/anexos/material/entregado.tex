En esta sección se describirá el material entregado junto a la presente documentación.  El material entregado se encuentra en un fichero comprimido y tiene el siguiente contenido:
\begin{itemize}
	\item \textbf{Mediciones receiver}  Este fichero contiene los resultados obtenidos de las mediciones realizadas durante las pruebas de rendimiento del Punto de Entrada de Datos.  El análisis de los resultados puede consultarse con detalle en la sección \ref{pruebas:rendimiento}.  Para una mayor comodidad del lector, los resultados de las mediciones también se encuentran en forma tabular en el anexo \ref{anexo:resultados_mediciones}.
	\item \textbf{Planificación inicial}  Este fichero contiene la planificación inicial del proyecto.  La planificación fue explicada con detalle en la sección \ref{planificacion:inicial}.
	\item \textbf{Planificación final}  Este fichero contiene la planificación real del proyecto.  La planificación fue explicada con detalle en la sección \ref{planificacion:real}.
	\item \textbf{Presupuestos}  Este fichero contiene los presupuestos de cliente y de costes para el proyecto.  Dichos presupuestos pueden consultarse en las secciones \ref{presupuesto:costes} y \ref{presupuesto:costes}.
	\item \textbf{Database}  Este fichero alberga los contenidos de la base de datos del LandPortal y puede ser utilizado para restaurar una copia de la base de datos sin necesidad de ejecutar los importadores\footnote{Los importadores se conectan a multitud de APIs y fuentes de datos externas para obtener los catálogos de datos que procesarán y enviarán al Punto de Entrada.  Debido al número de fuentes externas a consultar y al volumen de datos, este proceso puede durar varias horas.}.  El fichero se encuentra comprimido debido a que, por el gran volumen de datos que se maneja, el mismo fichero sin compresión ocupa 200MB.
	\item \textbf{Fuentes}  Este directorio contiene los ficheros de código fuente del sistema construido.
		\begin{itemize}
			\item \textbf{Installation scripts}  Este directorio contiene los \textit{scripts} de instalación del sistema.  Estos \textit{scripts} se han utilizado con éxito para desplegar el sistema en producción
			\item \textbf{Landportal model\footnote{\url{https://github.com/weso/landportal-drupal}}}  Este directorio contiene el código del modelo de datos del nuevo Land Portal.  Este modelo de datos puede consultarse con detalle en la sección \ref{modelo_datos_zona_datos}.
			\item \textbf{Landportal drupal\footnote{\url{https://github.com/weso/landportal_model}}}  Este directorio contiene el tema y los módulos de Drupal desarrollados durante este proyecto.  El diseño de estos componentes puede consultarse en las secciones \ref{vista_modulos_cms} y \ref{vista_landportal_uris}.
			\item \textbf{Landportal receiver\footnote{\url{https://github.com/weso/landportal-receiver}}}  Este directorio contiene el Punto de Entrada de Datos del sistema.  El diseño de este componente fue explicado anteriormente en la sección \ref{vista_receiver}.
		\end{itemize}	
	\item \textbf{Otros}  Este directorio contiene varios ficheros relacionados con el proyecto:
		\begin{itemize}
			\item \textbf{IFAD 2013 016 RFQ Technical Specifications}  Este fichero contiene las especificaciones técnicas del proyecto \textit{Rebuilding IFAD's LandPortal}, del cual forma parte el presente Trabajo Fin de Grado.
			\item \textbf{SBC4D Technical Offer}  Este fichero contiene la oferta técnica propuesta por la consultora SBC4D y bajo al cual se ha realizado este proyecto.
		\end{itemize}
\end{itemize}