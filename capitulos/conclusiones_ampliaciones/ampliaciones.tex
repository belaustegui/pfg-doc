A continuación se enumerarán las posibles ampliaciones que se podrían realizar
para aumentar el valor y la funcionalidad del sistema.


\subsection{Incluir soporte a nuevas visualizaciones en el subsistema de datos}
	El framework de soporte a visualizaciones podría ser ampliado para facilitar
	la creación de nuevos tipos de visualizaciónes con las que enriquecer la
	sección del portal dedicada a los datos.

\subsection{Incluir soporte a nuevas redes sociales para iniciar sesión en el sistema}
	Actualmente el sistema soporta el inicio de sesión utilizando las redes sociales
	Twitter y Facebook.  Permitir el inicio de sesión en el portal utilizando nuevas
	redes sociales como Google+, LinkedIn o GitHub facilitaría el acceso a los
	usuarios.

\subsection{Facilitar el mecanismo de obtención de claves de acceso al API}
	Actualmente los usuarios que quieran obtener una clave de acceso al API
	deben enviar un correo electrónico a la administración del portal, para que
	sea el administrador quien modifique el rol del usuario y le permita obtener
	su clave de aceso al API.
	
	Una ampliación interesante para los usuarios 
	consistiría en permitirles solicitar una nueva clave de acceso al API
	de forma automática desde la vista de su perfil de usuario.

\subsection{Gestión automática de los debates}
	Actualmente es la administración del portal quien se encarga de abrir y cerrar
	los debates cuando se alcance su periodo de participación.
	
	Una ampliación
	destinada a gestionar automáticamente la apertura y el cierre de los debates
	facilitaría la tarea de los administradores, sobre todo cuando el número de
	debates existentes es muy alto.

\subsection{Migración del contenido desde el viejo LandPortal}
	El viejo LandPortal cuenta con una gran cantidad de información creada por sus
	usuarios a lo largo del tiempo.  Una posible ampliación consistiría en migrar
	dicha información desde el viejo al nuevo portal.
	
	La tésis de Alan Chavoshe, titulada ``Linked Data for the Land Portal''
	\cite{lod_landportal}, contiene una explicación detallada de la estructura con la
	que se almacenan los datos en el viejo Land Portal.	Atendiendo a lo expuesto 
	en dicha tésis, el modelo de datos del viejo Land Portal difiere bastante
	del modelo utilizado en éste proyecto.  
	
	El primer paso para realizar ésta migración requeriría crear un nuevo módulo
	que extienda del módulo \textit{migrate}\footnote{El módulo \textit{migrate}
	es un módulo de Drupal destinado a facilitar la migración de contenidos 
	entre distintos portales.  El módulo \textit{migrate} esta disponible en
	\url{https://www.drupal.org/project/migrate}}.
	
	El mayor problema provendría de los tipos de contenido que no existen en 
	el nuevo portal como:  \textit{OrganisationType}, \textit{GPL Drivers} o
	\textit{Expertise}.  Para migrar éste tipo de contenidos sería necesario
	modificar la estructura de datos del nuevo Land Portal y modificar también
	las vistas correspondientes para presentarlos a los usuarios.
	
	Una primera estimación sobre el tiempo necesario para realizar ésta ampliación
	sería entre 100 o 150 horas, en las que modificar las estructuras del modelo
	de datos y las vistas necesarias.
	
\subsection{Futuros proyectos con el IFAD}
	Además de las posibles ampliaciones detalladas anteriormente, también existen una
	serie de futuros proyectos que se realizarán con el mismo cliente.
	\begin{itemize}
		\item Un \textit{hackatón}\footnote{Un \textit{hackatón} o \textit{hackathon} es un encuentro de programadores destinado a realizar un desarrollo colaborativo de software} para la creación de nuevos componentes del portal.  Concretamente
		el \textit{hackatón} consistiría en crear nuevos importadores para incluir catálogos de
		datos en el portal y nuevas visualizaciones con las que presentar los datos
		ya existentes.\\
		Originalmente el hackatón ha sido planteado para tener lugar durante el verano
		de 2014 en Alemania.
		
		\item Creación de la biblioteca o LandLibrary.  El LandLibrary es la tercera
		sección del portal y tiene como objetivo ser un repositorio de documentos,
		publicaciones, estudios, mapas, etc.  El LandLibrary tendrá también	capacidad
		de búsqueda y contará con un enriquecimiento semántico de los metadatos de
		las publicaciones que albergue.
		
		La información almacenada en el LandLibrary se hará pública en forma de
		datos enlazados abiertos para ser utilizados por servicios de terceros y 
		para ser incluidos en diferentes partes del portal, concretamente en el
		LandBook.
		
		Por su gran tamaño, ésta ampliación tendrá lugar como un proyecto con
		entidad propia.  Éste proyecto está planificado para comenzar en otoño
		del 2014.
		
	\end{itemize}

