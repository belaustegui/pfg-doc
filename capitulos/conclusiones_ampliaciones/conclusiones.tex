Una vez finalizado el sistema es el momento de realizar una retrospectiva y extraer
conclusiones, tanto técnicas como personales.

En primer lugar, se han adquirido multitud de conocimientos en nuevos lenguajes y 
frameworks que nunca había utilizado antes de éste proyecto.  En concreto, el 
conocimiento de un nuevo lenguaje de programación no se limita únicamente al 
conocimiento de su sintaxis, si no que también incluye el conocimiento
de sus conceptos (por ejemplo: \textit{duck typing} en Python frente a tipado fuerte en Java)
y abstracciones (por ejemplo: \textit{list comprehensions} y \textit{generator expressions} en Python frente a bucles \textit{for}
y \textit{foreach} en PHP), y la forma de combinarlas para conseguir un resultado
adecuado.

También se ha comprendido la necesidad de evitar caer en la optimización prematura
durante el desarrollo.  La optimización prematura causa en multitud de ocasiones 
un aumento en la complejidad del código y, por tanto, en el riesgo de introducir nuevos
errores.  En relación con ésto, también se ha aprendido a utilizar un
\textit{profiler} para obtener distintas métricas objetivas sobre el funcionamiento del
programa y analizar así qué puntos requieren una optimización y que puntos 
funcionan adecuadamente.

El tamaño del sistema a desarrollar ha puesto de manifiesto los beneficios del
uso de una buena arquitectura que permita desacoplar los diferentes componentes.
Éste desacoplamiento entre los diferentes componentes ha facilitado su desarrollo
y hace también posible el remplazo de un componente por otro con mínimos cambios
en el resto del sistema.

Siendo éste el primer proyecto en el que participo como parte de un equipo de 
desarrollo y con un cliente real al que se destina el proyecto, ha sido muy
importante aprender a trabajar en equipo y a tomar decisiones sobre el sistema
de forma conjunta.  En relación con ésto, también ha sido importante tomar
responsabilidad sobre diversas partes del sistema y comprender que de su correcto
funcionamiento depende el trabajo del resto de miembros del equipo.

Dado que, como se ha dicho antes, éste es mi primer proyecto con un cliente real,
ha sido también necesario aprender a trabajar manteniendo hitos inamovibles 
(o \textit{deadlines}) y adoptando requisitos cambiantes.\\
En varias ocasiones durante el desarrollo se han tenido que tomar decisiones
para poder alcanzar un determinado hito, ésto me ha introducido al concepto de
\textit{deuda técnica}\footnote{El concepto de \textit{deuda técnica} fue creado por
Ward Cunningham para explicar el coste producido en el sistema por aquellas situaciones
en las que es necesario adelantar trabajo (aunque no sea de la forma más correcta) y
arreglar lo ya hecho (pagar la deuda técnica) posteriormente.  Para más información
sobre éste concepto se recomienda el artículo de Martin Fowler al respecto \cite{mfowler:technical_debt}},
sus beneficios e inconvenientes y la forma de utilizarlo en las situaciones que sea necesario.


En resumen, éste proyecto ha supuesto multitud de cambios para mí: desde cambios en
las herramientas y metodologías de desarrollo hasta cambios en la organización y
coordinación del trabajo.  Formar parte de un equipo de trabajo en el que todos
los compañeros superan mis conocimientos ha sido una gran lección tanto técnica como
personal, y aprender a aprovechar esa situación para crecer y mejorar en ambos sentidos
ha sido, sin ningún tipo de duda, un acierto.