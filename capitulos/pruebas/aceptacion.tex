A continuación se muestran los reportes obtenidos del cliente durante las pruebas de aceptación\footnote{Únicamente se incluyen los reportes que afectan a las partes desarrolladas en este TFG.  La lista con todos los reportes obtenidos puede consultarse en \url{http://goo.gl/GhBDRw}}.  La categorización filtrado de los reportes ha sido realizada por el jefe de proyecto.

Todos los reportes que se listan a continuación han podido reproducirse correctamente y se encuentran arreglados en la versión final del sistema.
\begin{itemize}
	\item En los debates y las entradas del blog sería conveniente mostrar el nombre real del usuario en lugar de su \textit{nickname}.
	\item Cuando un registro se completa adecuadamente el sistema no muestra ningún mensaje al usuario tras hacer la redirección a la página de inicio.
	\item En el menú de la sección social se debe modificar la entrada de los debates por ``Discussions | Diálogos | Discussions''.
	\item Cuando un usuario solicita una nueva contraseña el sistema debería mostrar un mensaje indicando que la contraseña se ha restaurado correctamente antes de hacer la redirección a la página de inicio.
	\item El nombre de los usuarios que se muestra en las entradas de la sección social debería estar enlazado al perfil del usuario.
	\item Cuando se pulsa en una etiqueta de la zona social en ocasiones se muestra una vista errónea y similar a la de un nodo.
	\item Los enlaces para compartir contenidos en redes sociales como Twitter o Facebook contienen muchos espacios en el contenido.
	\item En los debates y las entradas del blog se muestra el texto ``\textit{What the user say}''.  Este texto es incorrecto y debería modificarse por ``\textit{What the users say}''.
	\item En la vista de eventos el nombre del calendario se muestra incorrectamente y en ocasiones se tapa por los botones de navegación.
	\item Las etiquetas ``\textit{Coming soon}'', ``\textit{Open}'' y ``\textit{Closed}'' de los debates no se muestran internacionalizadas.
	\item No existe ningún botón para que un usuario cierre sesión en el portal.
	\item Los \textit{combobox} de países muestran los países desordenados y con algunos espacios en blanco.  Han de mostrarse ordenados alfabéticamente según su nombre.
	\item Los \textit{combobox} de indicadores muestran los indicadores desordenados.  Han de mostrarse ordenados alfabéticamente según su nombre.
	\item Los indicadores que se muestran en la vista de detalles de un país deberían incluir la fuente de datos de la que provienen.
	\item El modelo de país no devuelve los indicadores del gráfico \textit{spider} si no tienen observaciones.  Esto provoca que en el gráfico las etiquetas se muestren vacías.
	\item En la zona social los debates deberían figurar como primera pestaña para darles una mayor visibilidad, puesto que son el punto central de participación de los usuarios.
	\item Las etiquetas que contienen el estado de los debates deberían estar coloreadas de forma diferente según su estado.  Ésto permite a los usuarios distinguir más fácilmente el estado en que se encuentra un debate.
	\item En el perfil de usuario se debería incluir una sección en la que se muestren todos los contenidos creados por el usuario.
	\item Los usuarios deberían incluir un campo en el que especificar sus intereses.
	\item En los formularios de la zona social se debería incluir una pequeña ayuda sobre las etiquetas HTML válidas, el comportamiento de los enlaces, etc.
\end{itemize}
