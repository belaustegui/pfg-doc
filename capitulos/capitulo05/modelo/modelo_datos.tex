\subsubsection{Diagrama del modelo de datos}
La figura \ref{fig:diagrama_modelo_final} muestra el diagrama con el diseño final del modelo del subsistema de datos.
\begin{landscape}
	\begin{figure}[h]
		\centering
		\includegraphics[width=25cm]{arquitectura/modelo}
		\caption{Diseño final del modelo de datos}
		\label{fig:diagrama_modelo_final}
	\end{figure}
\end{landscape}

Como se puede comprobar, existe una variación importante respecto a lo originalmente diseñado en la figura \ref{fig:clases_preliminares_modelo_book} perteneciente a la sección ``\nameref{clases_preliminares_modelo_datos}'' del capítulo \ref{chapter04}.  A continuación se expondrán las razones para ésta divergencia:

\paragraph{Especificación más cercana al RDF Data Cube Vocabulary} \hfill \\
Como se explicó anteriormente en el capítulo \ref{chapter03}, el RDF Data Cube Vocabulary organiza las observaciones en torno a un grupo de \textit{dimensiones}, \textit{medidas} y \textit{atributos}.

Para dar soporte a ésto se han incluido la clase ``\textit{Dimension}'', de la que heredarán todas las dimensiones, las clases ``\textit{MeasurementUnit}`` y ``\textit{Value} pretenden dar entidad completa a las medidas y, por último, las clases ``\textit{Computation}'' y ``\textit{Time}'' pretenden dar entidad a los atributos de las observaciones.

También se ha incluido soporte a slices con la clase ``\textit{Slice}''.

\paragraph{Inclusión de soporte para la internacionalización de datos} \hfill \\
Tal y como se definió en el requisito \textit{RLB 7} en la sección ``\nameref{requisitos_seccion_datos}'' del capítulo \ref{chapter04}, es necesario soportar la internacionalización en el nivel de datos.

Éste soporte ha sido incluido con las clases ``\textit{Language}``, ``\textit{OrganizationTranslation}``, ``\textit{RegionTranslation}``, ``\textit{TopicTranslation}`` e ``\textit{IndicatorTranslation}``.

\paragraph{Soporte a distintos intervalos temporales} \hfill \\
Puesto que las observaciones pueden hacer referencia a diferentes intervalos temporales, se han incluido las clases ``\textit{Interval}'', ``\textit{YearInterval}'' y ``\textit{MonthInterval}''.

\subsubsection{Elementos del modelo de datos}
A continuación se explicará el cometido de cada una de las clases del modelo que se han mostrado en la figura \ref{fig:diagrama_modelo_final}.

\begin{description}
	\item[Language]  Representa un idioma en el que se traducirán los datos.  Sus atributos son los siguientes:
		\begin{itemize}
			\item \textbf{Lang code}  Código de dos letras del idioma
			\item \textbf{Name}  Nombre del idioma
		\end{itemize}
	\item[RegionTranslation]  Traduce el nombre de una región.  Sus atributos son los siguientes:
		\begin{itemize}
			\item \textbf{Lang code}:  Código del idioma en el que se almacena la traducción
			\item \textbf{Region id}:  Identificador único de la región para la cual se almacena la traducción
			\item \textbf{Name}:  Nombre traducido de la región en el idioma correspondiente
		\end{itemize}
	\item[User]  Representa a un usuario que incluye información en el sistema (un importador).  Sus atributos son los siguientes:
		\begin{itemize}
			\item \textbf{Id}:  Identifica de forma única a cada usuario
			\item \textbf{IP}:  Dirección IP desde la que se realiza la petición de importación de datos
			\item \textbf{Timestamp}:  Momento temporal en el que se recibe la petición de importación de datos
			\item \textbf{Organization id}:  Identificador de la organización a la que pertenece el importador
		\end{itemize}
	\item[Organization]  Representa a una organización de la cual se incluyen datos en el sistema.  Sus atributos son los siguientes:
		\begin{itemize}
			\item \textbf{Id}:  Identifica de forma única a cada organización
			\item \textbf{Name}:  Es el nombre de la organización
			\item \textbf{URL}:  Dirección del sitio web de la organización
			\item \textbf{Is part of id}:  Identificador de la organización a la que pertenece (en el caso de que pertenezca a alguna)
		\end{itemize}
	\item[DataSource]  Representa una fuente de datos que aporta información al sistema.  Sus atributos son los siguientes:
		\begin{itemize}
			\item \textbf{Id}:  Identifica de forma única a cada fuente de datos
			\item \textbf{Name}:  Es el nombre de la fuente de datos
			\item \textbf{Organization id}:  Identificador de la organización a la que pertenece la fuente de datos
		\end{itemize}
	\item[DataSet]  Representa un conjunto de datos que se ha incluido en el sistema.  Sus atributos son los siguientes:
		\begin{itemize}
			\item \textbf{Id}:  Identifica de forma única a cada organización
			\item \textbf{Sdmx frequency}:  Frecuencia con la cual la fuente actualiza sus datos.  Los valores para este campo son tomados de la ontología SDMX\footnote{Para más información véase \url{http://publishing-statistical-data.googlecode.com/svn/trunk/specs/src/main/vocab/sdmx-code.ttl}}
			\item \textbf{Datasource id}:  Identificador de la fuente de datos a la que pertenece éste conjunto de datos
			\item \textbf{License id}:  Identificador de la licencia bajo la que se publica éste conjunto de datos
		\end{itemize}
	\item[OrganizationTranslation]  Almacena los datos de una organización en diferentes idiomas.  Sus atributos son los siguientes:
		\begin{itemize}
			\item \textbf{Lang code}:  Código del idioma en el que se almacena la traducción
			\item \textbf{Organization id}:  Identificador único de la organización para la cual se almacena la traducción
			\item \textbf{Description}:  Descripción traducida de la organización en el idioma correspondiente
		\end{itemize}
	\item[Slice]  Representa una slice\footnote{El concepto de slice fue explicado en la sección ``\nameref{concept:rdf_data_cube}'' del capítulo \ref{chapter03}}.  Sus atributos son los siguientes:
		\begin{itemize}
			\item \textbf{Id}:  Identificador único de la slice
			\item \textbf{Indicator id}:  Identificador único del indicador con el cual se relaciona el slice
			\item \textbf{Dimension}:  Identificador único de la otra dimensión con la que se relaciona el slice (o bien una región o bien un momento temporal)
			\item \textbf{Dataset id}:  Identificador único del conjunto de datos del que procede la slice
		\end{itemize}
	\item[Observation]  Representa una observación para un indicador y una región en un determinado momento temporal.  Sus atributos son los siguientes:
		\begin{itemize}
			\item \textbf{Id}:  Identificador único de la observación
			\item \textbf{Ref time id}:  Identificador único del momento temporal al que la observación hace referencia
			\item \textbf{Issued id}:  Identificador único del momento temporal en el que la observación fue incluida en el sistema
			\item \textbf{Computation id}:  Identificador único de la computación con la que se relaciona la observación
			\item \textbf{Indicator group id}:  Identificador único del grupo de indicadores con el que se relaciona la observación (si se relaciona con alguno)
			\item \textbf{Value id}:  Identificador único del valor de la observación
			\item \textbf{Indicator id}:  Identificador único del indicador al que pertenece la observación
			\item \textbf{Dataset id}:  Identificador único del conjunto de datos al que pertenece la observación
			\item \textbf{Region id}:  Identificador único de la región a la que hace referencia la observación
			\item \textbf{Slice id}:  Identificador único del slice al que pertenece la observación (en caso de pertencer a alguno)
		\end{itemize}
	\item[Indicator]  Representa un indicador sobre el cual se realizan mediciones.  Sus atributos son los siguientes:
		\begin{itemize}
			\item \textbf{Id}:  Identificador único del indicador
			\item \textbf{Preferable tendency}:  Representa la tendencia preferible del indicador a lo largo del tiempo
			\item \textbf{Measurement unit id}:  Identificador único de la unidad de medida para los valores de las observaciones del indicador
			\item \textbf{Compound indicator id}:  Identificador del indicador compuesto con el que se relaciona el indicador (en caso de relacionarse con alguno)
			\item \textbf{Last update}:  Fecha de la ultima actualización del indicador o alguna de sus observaciones en el sistema
			\item \textbf{Starred}:  Indica si el indicador ha sido marcado como favorito o no
			\item \textbf{Topic id}:  Identificador único del tópico al que pertenece el indicador
		\end{itemize}
	\item[IndicatorTranslation]  Almacena los datos de un indicador en diferentes idiomas.  Sus atributos son los siguientes:
		\begin{itemize}
			\item \textbf{Lang code}:  Código del idioma en el que se almacena la traducción
			\item \textbf{Indicator id}:  Identificador único del indicador para el cual se almacena la traducción
			\item \textbf{Name}:  Nombre del indicador traducido en el idioma correspondiente
			\item \textbf{Description}:  Descripción del indicador traducida en el idioma correspondiente
		\end{itemize}
	\item[Topic]  Representa un tópico sobre el cual se guardan indicadores.  Su único atributo es un código que lo identifica.
	\item[TopicTranslation]  Almacena los datos de un tópico en diferentes idiomas.  Sus atributos son los siguientes:
		\begin{itemize}
			\item \textbf{Lang code}:  Código del idioma en el que se almacena la traducción
			\item \textbf{Indicator id}:  Identificador único del tópico para el cual se almacena la traducción
			\item \textbf{Name}:  Nombre del tópico traducido en el idioma correspondiente
		\end{itemize}
	\item[IndicatorGroup]  Representa una agrupación de indicadores compuestos.  Su único atributo es un código que lo identifica.
	\item[MeasurementUnit]  Representa a una unidad de medida.  Sus atributos son los siguientes:
		\begin{itemize}
			\item \textbf{Id}:  Identificador único de la unidad de medida
			\item \textbf{Name}:  Nombre de la unidad de medida
			\item \textbf{ConvertibleTo}:  Unidad estándar a la que es convertible\footnote{Por ejemplo: si la unidad de medida fueran centímetros, la unidad estándar a la que es convertible serían metros}
			\item \textbf{Factor}:  Factor de conversión a aplicar al valor para convertirlo a la unidad estándar
		\end{itemize}
	\item[License]  Representa una licencia bajo la cual se publica algún conjunto de datos almacenado en el sistema.  Sus atributos son los siguientes:
		\begin{itemize}
			\item \textbf{Id}:  Identificador único de la licencia
			\item \textbf{Name}:  Nombre de la licencia
			\item \textbf{Description}:  Descripción larga de la licencia
			\item \textbf{Republish}:  Indica si la licencia autoriza o no a republicar los datos
			\item \textbf{URL}:  Sitio web de la licencia
		\end{itemize}
	\item[Computation]  Representa una computación que se realiza sobre los valores de una observación.  Sus atributos son:
		\begin{itemize}
			\item \textbf{Id}:  Identificador único de la computación
			\item \textbf{Name}:  URL de la computación\footnote{Las URL de las computaciones son tomadas de la ontología WESO-Computex, para más información véase \url{https://raw.githubusercontent.com/weso/computex/master/ontology/computex.rdf}}
			\item \textbf{Description}:  Descripción de la computación
		\end{itemize}
	\item[Value]  Representa un valor de una observación.  Sus atributos son:
		\begin{itemize}
			\item \textbf{Id}:  Identificador único del valor
			\item \textbf{Obs status}:  Estado de la observación\footnote{Éste campo permite distinguir entre aquellas observaciones nulas y aquellas observaciones que no cuentan con un valor.  Ésta distinción es necesaria para asegurar la calidad estadística de los datos.}
			\item \textbf{Value}:  Almacena el valor concreto
			\item \textbf{Value type}:  Indica el tipo del valor almacenado
		\end{itemize}
	\item[IndicatorRelationship]  Representa una relación entre varios indicadores.  Su único atributo es un código que identifica la relación.  Existen dos tipos de relaciones:  \textbf{Becomes}, que representa un indicador que se trasnforma en otro a lo largo del tiempo y \textit{IsPartOf}, que representa un indicador que forma parte de otro
	\item \textbf{Dimension}:  Representa una dimensión tal y como se define en el \nameref{concept:rdf_data_cube}.  Su único atributo es un código que la identifica
	\item[Time]  Representa una unidad de tiempo.  Hereda los atributos de \textbf{Dimension}
	\item[Instant]  Representa un momento concreto en el tiempo.  Además de los atributos que hereda de \textbf{Time} también incluye:
	\begin{itemize}
		\item \textbf{Timestamp}:  Momento concreto en el tiempo al que hace referencia
	\end{itemize}
	\item[Interval]  Representa un intervalo arbitrario de tiempo.  Además de los atributos que hereda de \textbf{Time} también incluye:
	\begin{itemize}
		\item \textbf{Start time}:  Momento en el que comienza el intervalo de tiempo
		\item \textbf{End time}:  Momento en el que finaliza el intervalo de tiempo
		\item \textbf{Value}:  Valor del intervalo en forma de texto para su posterior presentación en las vistas
	\end{itemize}
	\item[YearInterval]  Representa un intervalo de un año.  Además de los atributos que hereda de \textbf{Interval} también incluye:
	\begin{itemize}
		\item \textbf{Year}:  Año concreto al que hace referencia el intervalo
	\end{itemize}
	\item[MonthInterval]  Representa un intervalo de un mes.  Además de los atributos que hereda de \textbf{Interval} también incluye:
	\begin{itemize}
		\item \textbf{Year}:  Año concreto al que hace referencia el intervalo
		\item \textbf{Month}:  Mes concreto al que hace referencia el intervalo
	\end{itemize}
	\item[Region]  Representa una región sobre la que se almacenan observaciones.  Además de los atributos que hereda de \textbf{Dimension} también tiene:
	\begin{itemize}
		\item \textbf{Un code}:  Será el código numérico asignado por la División Estadística de las Naciones Unidas\footnote{Éste código recibe el nombre formal de ``ISO 3166-1 numeric''.  En \cite{un:standard-country-codes} y \cite{un:iso-3166-country-codes} puede encontrarse más información sobre éstos códigos.} para la región
		\item \textbf{Is part of id}:  Identificador único de la región de la que forma parte (en caso de formar parte de alguna)
	\end{itemize}
	\item[Country]  Representa un país sobre el que se almacenan observaciones.  Además de los atributos que hereda de \textbf{Region} también tiene:
	\begin{itemize}
		\item \textbf{Fao URI}:  URI única del país en la Ontología Geopolítica de la Organización para la Alimentación y la Agricultura de las Naciones Unidas\footnote{La información sobre ésta ontología se encuentra disponible en \cite{fao:geopolitical-ontology}} ``\nameref{clases_preliminares_modelo_datos}'' perteneciente al capítulo \ref{chapter04}
		\item \textbf{ISO2}:  Código alfabético de dos letras asignado por la división estadística de las Naciones Unidas\footnote{Éste código recibe el nombre formal de ``ISO 3166-1 alpha-2''}
		\item \textbf{ISO3}:  Código alfabético de tres letras asignado por la división estadística de las Naciones Unidas\footnote{Éste código recibe el nombre formal de ``ISO 3166-1 alpha-3''}
		\item \textbf{Taxonomy id}:  Identificador del país en la taxonomía creada por el gestor de contenidos
	\end{itemize}
	\item[CompoundIndicator]  Representa un indicador compuesto de varios indicadores simples.  Además de los atributos que hereda de \textbf{Indicator} también incluye:
		\begin{itemize}
			\item \textbf{Indicator ref group id}:  Identificador único del grupo de indicadores al que pertenece éste indicador compuesto
		\end{itemize}
\end{description}

