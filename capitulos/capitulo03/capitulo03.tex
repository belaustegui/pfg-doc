\section{Datos abiertos}
El concepto de datos abiertos u \textit{Open Data} es definido por la Open Knowledge Foundation en \cite{opendefinition} de la siguiente forma:
\begin{quote}
\textit{``A piece of data or content is open if anyone is free to use, reuse, and redistribute it — subject only, at most, to the requirement to attribute and/or share-alike.''}
\end{quote}

De la anterior definición pueden destacarse tres pilares fundamentales que marcan la diferencia entre los datos abiertos y el resto de datos, dichos puntos clave se explicarán a continuación:
\begin{itemize}
\item Acceso y disponibilidad. Los datos deben estar disponibles en su totalidad con un coste razonable o, preferiblemente, de forma gratuita a través de Internet.  Es también importante que el formato en el que se publican los datos sea conveniente y modificable.
\item Reutilización y redistribución.  La licencia bajo la que se publican los datos no debe restringir la redistribución de los mismos ni exigir pagos o cuotas por dicha redistribución.  Además, la licencia debe permitir modificar los datos e incluso cruzarlos con datos provenientes de otras fuentes.
\item Participación universal.  Los datos no deben poner trabas para ser accedidos por ninguna persona ni grupo de personas ni deben restringir el uso de los mismos a un ámbito de trabajo específico.
\end{itemize}

Cualquier conjunto de datos puede ser considerado un conjunto de datos abiertos si cumple con la definicion anterior, aunque las principales fuentes de datos abiertos generalmente provienen de fuentes científicas o gubernamentales.  En el año 2004 los ministros de ciencia de todas las naciones pertenecientes a la OECD\footnote{La OECD (Organización para la Cooperación y el Desarrollo Económico) es una de las organizaciones que aporta los conjuntos de datos con los que se trabaja en éste proyecto.} firmaron una declaración en la que se aboga por hacer públicos toda la información científica financiada con fondos públicos.

En cuanto a los datos abiertos procedentes de fuentes gubernamentales, como ya se ha mencionado en el capítulo \ref{chapter01} - \nameref{chapter01}, la tendencia de los gobiernos de ofrecer datos en forma abierta es cada vez mayor tal y como evidencian los múltiples portales de datos propiedad de la administración pública que han surgido recientemente. Algunos ejemplos son: el portal de datos del Gobierno de España\footnote{http://datos.gob.es/}, el portal de datos del Gobierno de Reino Unido\footnote{http://data.gov.uk/} o el portal de datos del Gobierno de Estados Unidos\footnote{http://www.data.gov/}.



\section{Datos enlazados}
El concepto de datos enlazados o \textit{Linked Data} fue acuñado por Tim Berners-Lee, director del W3C\footnote{World Wide Web Consortium}, y describe un método de publicar catálogos datos estructurados de una forma en la que sea posible conectarlos con otros catálogos de datos procedentes de diferentes fuentes.

En \cite{tbl-linkedopendata}, Tim Berners-Lee define los cuatro elementos fundamentales para los datos enlazados:
\begin{itemize}
\item Usar URIs para nombrar los elementos.
\item Usar URIs HTTP para permitir a las personas acceder y buscar dichos nombres.
\item Cuando alguien accede a una URI, devolver la información de forma útil usando los estándares RDF o SPARQL.
\item Incluir enlaces a otras URIs para facilitar el descubrimiento de más elementos.
\end{itemize}


\section{Datos enlazados abiertos}
El concepto de datos enlazados abiertos (\textit{Linked Open Data}) combina los conceptos de datos abiertos (\textit{Open Data}) y datos enlazados (\textit{Linked Data}) que se han explicado en las secciones anteriores.

Un conjunto de datos se considera un conjunto de datos enlazados abiertos si cumple los requisitos de los datos enlazados y además se presenta bajo una licencia abierta que no impida su reutilización ni redistribución.


\subsection{Sistema de estrellas}
Como se menciona en \cite{tbl-linkedopendata}, en el año 2010 Tim Berners-Lee desarrolló un sistema de estrellas o niveles con el objetivo de concienciar (principalmente a las organizaciones gubernamentales) en el uso de datos enlazados abiertos.

A continuación se explica el significado de cada nivel de esta escala:
\begin{enumerate}
    \item Los datos están disponibles en cualquier formato (pero manteniendo una licencia libre para ser considerados datos abiertos).  Un ejemplo serían datos publicados como una imagen escaneada.
    \item Los datos se encuentran en un formato estructurado y que pueda ser leído por máquinas.  Por ejemplo utilizar un formato \textit{Microsoft Excel} en lugar de una imagen escaneada.
    \item Similar al anterior pero utilizando un formato libre, por ejemplo CSV, JSON o XML.
    \item Cumple con todos los niveles anteriores pero además utiliza un estándar abierto de la W3C como RDF o SPARQL.  El uso de estos estándares permite que el catálogo de datos sea enlazado por otros catálogos u organizaciones.
    \item Cumple con todos los niveles anteriores, pero además enlaza hacia otros catálogos de datos externos.
\end{enumerate}

El portal de datos que se pretende construir en éste proyecto pretende cumplir con un nivel de 5 estrellas.


