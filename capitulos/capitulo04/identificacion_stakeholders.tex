El PMBOK \cite{pmi:pmbok} define los \textit{stakeholders} de un proyecto como aquellas personas u organizaciones que:
\begin{itemize}
\item Están directamente involucradas en el proyecto
\item Tienen intereses que pueden verse afectados por la correcta ejecución del proyecto
\item Pueden influir a la marcha del proyecto, a los entregables o a los miembros del equipo
\end{itemize}

En esta sección se identificarán los \textit{stakeholders} tanto directos como indirectos que formarán parte del sistema.


\subsection{\textit{Stakeholders} directos}
Los stakeholders directos son aquellos que utilizarán directamente el sistema tras su finalización.  Para proveer un contexto adecuado de cara a las siguientes secciones del análisis del sistema se detallarán todas las características de los \textit{stakeholders} directos del nuevo Land Portal aunque no formen parte del presente proyecto.  Las características de los \textit{stakeholders} directos que sí forman parte de éste proyecto podrán verse con detalle a continuación en la sección de \nameref{requisitos_sistema}.

Existirán cinco tipos de \textit{stakeholders} que interactúan con el sistema, éstos  son: usuarios anónimos, usuarios registrados, usuarios con acceso al API, administradores e importadores de datos.
\begin{description}
\item[Usuarios anónimos]  Los usuarios anónimos son todos aquellos usuarios del portal que no hayan iniciado sesión con una cuenta de usuario.  El papel de estos usuarios en el sistema será el de consumidores de información, puesto que únicamente se les permitirá visualizar los contenidos de la zona de datos y de la zona social (exceptuando acceder a la información de los perfiles de otros usuarios registrados).  Éstos usuarios también podrán utilizar la búsqueda, registrar una nueva cuenta de usuario o iniciar sesión con una cuenta de usuario ya existente.
\item[Usuarios registrados]  Los usuarios registrados son todos aquellos usuarios que tienen una cuenta de usuario en el portal y que además han iniciado sesión con ella.  El papel de éstos usuarios será tanto de consumidores como creadores de información, puesto que además de ver los contenidos de la zona de datos y la zona social (incluyendo acceder a la información de los perfiles de otros usuarios registrados) también podrán aportar nueva información a la zona social.  Concretamente podrán crear debates, eventos y noticias, además de comentar en los debates abiertos o en las entradas del blog.
\item[Usuarios con acceso al API]  Los usuarios con acceso al API son usuarios registrados que además cuentan con una clave de acceso al API pública del portal.  Éstos usuarios jugarán un papel de creadores, consumidores y difusores de información, puesto que además de todas las capacidades de los usuarios registrados también tendrán un acceso total al API que podrán utilizar para crear servicios o aplicaciones externas que se beneficien de los datos ofrecidos por el nuevo Land Portal.
\item[Administradores]  Los administradores serán aquellos usuarios de confianza que se encarguen de mantener el funcionamiento del nuevo Land Portal.  Éstos usuarios tendrán principalmente un papel de moderadores de la zona social del portal.  Tendrán capacidad para editar o eliminar los debates, noticias o eventos creados por otros usuarios; abrir o cerrar los debates para que el resto de usuarios puedan participar en ellos; moderar o eliminar los comentarios introducidos por otros usuarios; publicar o modificar contenido en el blog de Land Portal; gestionar el contenido del catálogo de datos y otorgar o eliminar las capacidades de administración o acceso al API del resto de usuarios registrados.
\item[Importadores de datos]  A diferencia de los \textit{stakeholders} descritos anteriormente, los importadores de datos no serán personas, si no que serán aplicaciones creadas con el fin de insertar nuevos datos en el portal.  El objetivo de éstas herramientas será capturar datos provenientes de diferentes fuentes u organizaciones, transformar dichos datos a un XML Schema definido y enviarlos al punto de entrada de datos del portal.
\end{description}


\subsection{\textit{Stakeholders} indirectos}
Los \textit{stakeholders} indirectos son aquellos que están interesados y tendrán influencia en la marcha del proyecto y el sistema final pero que no lo utilizarán directamente.

\begin{description}
\item[Desarrolladores]  Los desarrolladores serán los encargados de construir el sistema final.  Los desarrolladores pertenecerán al grupo de investigación WESO.
\item[Jefes de proyecto]  Los jefes de proyecto serán los encargados de guiar y orientar el desarrollo del sistema y de interactuar con los clientes finales.  Los jefes de proyecto pertenecerán al grupo de investigación WESO y a la empresa SB Consulting.
\item[Cliente final]  El cliente final está interesado en el sistema pues es quien financia el proyecto y explotará el sistema una vez haya sido completado
\end{description}