Tal y como se ha mencionado en el capítulo de \nameref{chapter01}, éste proyecto se encuadra dentro del proyecto \textit{Rebuilding IFAD's LandPortal RFQ/2013/016/SC} desarrollado por el grupo de investigación WESO y la empresa SB Consulting. El cliente a quien va destinado es Fondo Internacional para el Desarrollo Agrícola, que forma parte de la Organización de las Naciones Unidas.
Conviene definir en este momento una terminología común que se va a utilizar de ahora en adelante.  Se hablará de \textit{sistema} para hacer referencia al nuevo Land Portal en su totalidad, por otra parte se hablará de \textit{proyecto} para hacer referencia al backend del nuevo Land Portal del que es objeto la presente documentación.


\subsection{Alcance del sistema}
A continuación se explicará qué partes del desarrollo del nuevo Land Portal tienen cabida en éste proyecto.  También se mencionarán las partes que no entrarán dentro del alcance del proyecto para que el lector pueda apreciar la complejidad del sistema en su totalidad.

\subsubsection{Elementos dentro del alcance del proyecto}
\begin{itemize}
\item Punto de entrada único para los datos del portal.  Como se ha explicado anteriormente los datos del portal procederán de diversas fuentes de datos y organizaciones externas.  Con el fin de mantener el control sobre el tiempo y forma en la que se incluyen nuevos datos será necesario crear un punto único de entrada para los mismos.  Para garantizar una cierta uniformidad y asegurar un nivel de calidad mínimo también será necesario definir un formato en el que enviar los datos hacia éste punto de entrada.\newline
Por otra parte, un componente como éste tiene una gran responsabilidad en el correcto funcionamiento del portal, por lo que será necesario establecer unas medidas de seguridad que eviten la introducción de datos procedentes de orígenes no confiables.\newline
Además será necesario insertar en una base de datos relacional los datos que lleguen al punto de entrada.  Los datos que se almacenen en ésta base de datos se utilizarán para la construcción del framework que da soporte a las visualizaciones de datos y que se verá a continuación.
\item Framework para proveer información con la que construir visualizaciones de datos.  Éste portal de datos no se limitará a almacenar y devolver catálogos de datos, si no que también ofrecerá visualizaciones que permitan a los usuarios acceder los datos de una forma sencilla y atractiva.  Como parte del proyecto se desarrollará un framework que permita proveer la información necesaria para construir las visualizaciones de datos.
\item Arquitectura para la creación de vistas personalizadas.  Además de la creación de un framework que provee la información necesaria para construir visualizaciones de datos también será necesario la creación de una arquitectura que permita incluir vistas personalizadas   Con el fin de hacer ésta arquitectura lo más general y reutilizable posible se evitará hacer uso de los mecanismos que el CMS provee para la creación de vistas.
\item Mecanismo de internacionalización.  Puesto que éste portal ofrecerá datos procedentes de diversas organizaciones internacionales y relativos a multitud de países y continentes distintos será necesario proveer un mecanismo de internacionalización que permita a los usuarios acceder a la información en el lenguaje que prefieran.  Además será necesario que el mecanismo de internacionalización no se limite simplemente a soportar traducciones para la información estática, si no que tendrá que ir más allá y permitir la internacionalización de los propios datos.
\item Plataforma social que fomente la participación de los usuarios y complemente la información de los conjuntos de datos.  Puesto que el nuevo Land Portal pretende hacer especial énfasis en la participación de los usuarios será necesario ofrecer una plataforma en la que la comunidad pueda interactuar e intercambiar información.  Bajo dicha plataforma se crearán debates para que los usuarios intercambien sus opiniones a cerca de algún tema concreto, noticias para mantener al resto de usuarios informados sobre aquellas informaciones que se consideren necesarias y eventos que tendrán lugar en una fecha concreta.  Además también albergará un blog en el que el propio Land Portal coloque aquella información que considere relevante para sus usuarios.\newline
Dado que ésta plataforma estará completamente integrada dentro del nuevo Land Portal será también necesario que cuente con un aspecto uniforme y que mantenga la línea de identidad del portal, de forma que la transición entre las diferentes partes sea transparente al usuario.  Las vistas realizadas para ésta parte del portal utilizarán los propios mecanismos ofrecidos por el CMS para la creación de vistas y plantillas visuales. 
\item Componente de autenticación de los usuarios para usar el API.  Como se verá posteriormente la implementación del API del nuevo Land Portal queda fuera del alcance de éste proyecto, aunque sí que será necesario implementar un componente que permita controlar las claves de acceso al API.  La seguridad del API es un apartado muy importante, por lo que sólo deberán tener acceso aquellos usuarios que la administración desee.  En relación con el punto anterior, la generación de las claves de acceso deberá realizarse de forma transparente al usuario.
\item Unificación de la búsqueda entre las diferentes partes del portal.  Con el fin de centralizar la búsqueda en una única parte del portal, será necesario unificar la búsqueda de la parte social perteneciente al CMS y de la parte de datos, cuyos datos se encuentran almacenados fuera del CMS.
\end{itemize}

\subsubsection{Elementos fuera del alcance del proyecto}
\begin{itemize}
\item Generación de RDF y enriquecimiento de datos.  Como se ha explicado anteriormente uno de los objetivos de éste proyecto es diseñar un punto de entrada único para los datos del portal, además de implementar un mecanismo que inserte dichos datos en una base de datos relacional.  Un elemento fuera del alcance de éste proyecto, pero que sí se encuentra dentro del sistema real es un componente encargado de la generación de datos en formato RDF y el enriquecimiento de los mismos.  Además dicho componente también almacenará los datos en Virtuoso, que fue escogido como servidor semántico tal y como se mencionó en la sección ``\nameref{chapter02:alternativas_seleccionadas}'' del capítulo \ref{chapter02}.
\item Importación de datos.  Siendo el sistema que se pretende construir un portal de datos, la importación de los propios datos juega un papel clave en la buena marcha del mismo.  Tal y como ya se ha explicado en la sección ``\nameref{objetivos_proyecto}'' perteneciente al capítulo \ref{chapter01} los datos con los que se trabajará procederán de varias y muy diversas fuentes.  La importación de datos consistirá en unificar todos esos datos en un formato común y enviarlos hacia el punto de entrada de datos para que puedan ser visualizados en el portal.
\item Creación de visualizaciones.  Como se ha explicado anteriormente queda dentro del alcance del proyecto la creación de un framework que provea la información necesaria para crear visualizaciones de datos.  Por la complejidad de las visualizaciones, éstas quedarán fuera del alcance del proyecto y serán implementadas por un diseñador con experiencia previa en el ambito de la visualización de datos.
\item Integración con el catálogo de datos.  En la sección ``\nameref{chapter02:alternativas_seleccionadas}'' perteneciente al capítulo \ref{chapter02} de ésta misma documentación, se indicó que se utilizará CKAN como catálogo de datos.  La integración de CKAN con el resto del portal quedará fuera del alcance del proyecto, así como la incorporación de los datos que llegan por el punto de entrada dentro el propio catálogo. 
\item Interfaz visual del portal de datos.  Anteriormente se ha mencionado que sí entrará dentro del alcance del proyecto la creación de una interfaz visual para la parte social del portal.  La interfaz visual del portal de datos quedará sin embargo fuera del alcance de éste proyecto por la propia necesidad de conseguir una estrecha relación con las visualizaciones de datos.  A diferencia de la interfaz visual perteneciente a la plataforma social, la interfaz del portal de datos utilizará la arquitectura para la creación de vistas personalizadas y evitará la utilización de los mecanismos ofrecidos por el CMS.
\end{itemize}