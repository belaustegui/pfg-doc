\subsection{Especificación de los requisitos funcionales}
A continuación se procederá a obtener, analizar y organizar los requisitos con los que contará el sistema que se construirá en éste proyecto.  Cabe destacar que éste listado de requisitos sólo recoge los requisitos que se implementarán en éste proyecto y es un subconjunto de todos los requisitos que forman parte del nuevo Land Portal.\newline
Para hacer más sencilla la lectura, el catálogo de requisitos se dividirá en diferentes tablas dependiendo del componente al que afecten.

Cada entrada de la tabla de requisitos contendrá la siguiente información:
\begin{itemize}
\item Código de identificación. El código de identificación pretende identificar a cada requisito de forma unívoca para hacer así posible referirse a él posteriormente.
\item Nombre del requisito. El nombre pretende introducir de forma corta y sencilla el objetivo de cada requisito.
\item Descripción del requisito. La descripción pretende detallar cada requisito en profundidad.
\end{itemize}

\subsubsection{Requisitos de la sección de datos}
\label{requisitos_seccion_datos}
De ahora en adelante la sección de datos del portal recibirá el nombre de ``LandBook''.  A continuación, en la tabla  \ref{requisitos_datos} se muestra el listado de requisitos pertenecientes al LandBook.
\begin{longtable}[c]{|p{1mm}|p{14mm}|p{30mm}|p{90mm}|}
 \caption{Tabla de requisitos de la sección de datos.\label{requisitos_datos}}\\

 %Cabecera en la primera pagina
 \hline
 \multicolumn{4}{| c |}{Listado de requisitos de la sección de datos}\\
 \hline
 \multicolumn{2}{|c|}{Código} & Nombre & Descripción\\
 \hline
 \hline
 \endfirsthead
 
 %Cabecera en el resto de páginas
 \hline
 \multicolumn{4}{|c|}{Continuación de la tabla \ref{requisitos_datos}}\\
 \hline
 \multicolumn{2}{|c|}{Código} & Nombre & Descripción\\
 \hline
 \hline
 \endhead
 
 \hline
 \endfoot
 

\multicolumn{2}{|l|}{RLB 1}  & Contenido de la sección de datos & La sección de datos estará compuesta de: regiones, países, indicadores, organizaciones, catálogo de datos y widgets. \\
\hline
\multicolumn{2}{|l|}{RLB 2}  & Acceso a la sección de datos & El sistema permitirá acceder al contenido de la sección de datos tanto a usuarios registrados como anónimos. \\
\hline
\multicolumn{2}{|l|}{RLB 3}  & Identificación del contenido & Todo el contenido ofrecido en el LandBook tendrá una URL única. \\
\hline
\multicolumn{2}{|l|}{RLB 4}  & Arquitectura para las vistas & Las vistas del LandBook deberán evitar el uso de los mecanismos del plantillas visuales del CMS. \\
\hline
 & RLB 4.1 & Arquitectura para las vistas & Las rutas para las nuevas vistas se especificarán a través de un fichero de configuración, de forma que sea posible modificarlas sin necesidad de acceder al código fuente. \\
\hline
 & RLB 4.2 & Arquitectura para las vistas & Las plantillas y modelos para las vistas se buscarán utilizando un mecanismo de convenio de nombres. \\
\hline
\multicolumn{2}{|l|}{RLB 5}  & Integración con la zona social & Cuando se acceda a un país en el LandBook, se mostrará un enlace para ver todo el contenido relacionado con dicho país creado en la zona social.. \\
\hline
\multicolumn{2}{|l|}{RLB 6}  & Soporte a las visualizaciones & El LandBook proveerá un sistema que pueda ser utilizado por las vistas con el fin de crear visualizaciones de datos. \\
\hline
& RLB 6.1 & Soporte a las visualizaciones & El sistema permitirá devolver el valor medio de todas las observaciones existentes para una región e indicador concretos. \\
\hline
& RLB 6.2 & Soporte a las visualizaciones & El sistema permitirá devolver todas las observaciones existentes para una región e indicador concretos. \\
\hline
& RLB 6.3 & Soporte a las visualizaciones & El sistema permitirá devolver el valor medio de todas las observaciones existentes para un determinado indicador, independientemente de la región a la que hagan referencia. \\
\hline
& RLB 6.4 & Soporte a las visualizaciones & El sistema permitirá devolver una comparación del el valor medio de todas las observaciones existentes para dos indicadores concretos, independientemente de la región a la que hagan referencia. \\
\hline
& RLB 6.5 & Soporte a las visualizaciones & El sistema permitirá devolver todas las observaciones para un país e indicador concretos. \\
\hline
\multicolumn{2}{|l|}{RLB 7}  & Soporte a la internacionalización & El LandBook permitirá mostrar la información que contiene en diferentes idiomas. \\
\hline
& RLB 7.1 & Soporte a la internacionalización & El LandBook permitirá acceder a la información en inglés. \\
\hline
& RLB 7.2 & Soporte a la internacionalización & El LandBook permitirá acceder a la información en francés. \\
\hline
& RLB 7.3 & Soporte a la internacionalización & El LandBook permitirá acceder a la información en español. \\
\hline
\hline

 \end{longtable}

\subsubsection{Requisitos de la sección social}
\label{requisitos_seccion_social}
De ahora en adelante la sección social del portal recibirá el nombre de ``LandDebate''.  En la tabla \ref{requisitos_debate} se muestra el listado de requisitos pertenecientes al LandDebate.
\begin{longtable}[c]{|p{1mm}|p{14mm}|p{30mm}|p{90mm}|}
 \caption{Tabla de requisitos de la zona social del portal.\label{requisitos_debate}}\\

 %Cabecera en la primera pagina
 \hline
 \multicolumn{4}{| c |}{Listado de requisitos de la búsqueda}\\
 \hline
 \multicolumn{2}{|c|}{Código} & Nombre & Descripción\\
 \hline
 \hline
 \endfirsthead
 
 %Cabecera en el resto de páginas
 \hline
 \multicolumn{4}{|c|}{Continuación de la tabla \ref{requisitos_debate}}\\
 \hline
 \multicolumn{2}{|c|}{Código} & Nombre & Descripción\\
 \hline
 \hline
 \endhead
 
 \hline
 \endfoot
 


\multicolumn{2}{|l|}{RLD 1} & Registro de usuarios & El LandDebate permitirá realizar registros de nuevos usuarios en el sistema. \\
\hline
& RLD 1.1 & Registro de usuarios & Los nuevos registros requerirán la introducción de un nombre de usuario.  Éste nombre de usuario será único en todo el portal. \\
\hline
& RLD 1.2 & Registro de usuarios & Los nuevos registros requerirán la introducción de una contraseña de usuario.  Para evitar errores será necesario que el usuario repita la contraseña antes de completar el registro. \\
\hline
& RLD 1.4 & Registro de usuarios & Los nuevos registros requerirán la introducción del nombre real y los apellidos de la persona que se está registrando. \\
\hline
& RLD 1.5 & Registro de usuarios & Los nuevos usuarios podrán seleccionar el continente en el que se encuentran.  Éste campo no será obligatorio. \\
\hline
& RLD 1.6 & Registro de usuarios & Los nuevos usuarios podrán seleccionar hasta un máximo de 7 países en los que estén interesados.  Éste campo no será obligatorio. \\
\hline
& RLD 1.7 & Registro de usuarios & El registro de usuarios será accesible desde cualquier punto del portal. \\
\hline
& RLD 1.8 & Registro de usuarios & Los usuarios podrán registrarse en el portal utilizando sus cuentas de Twitter o Facebook. \\
\hline
\multicolumn{2}{|l|}{RLD 2} & Roles de usuario & Los usuarios del portal tendrán diferentes roles de usuario en función de los cuales podrán realizar diferentes tareas en el LandDebate. \\
\hline
& RLD 2.1 & Roles de usuario & El rol de usuario \textit{anónimo} será automáticamente asignado a todos los usuarios que no se hayan registrado o no hayan iniciado sesión en el portal. \\
\hline
& RLD 2.2 & Roles de usuario & El rol de usuario \textit{registrado} será automáticamente asignado a todos los usuarios que se hayan registrado e inicien sesión en el portal. Los usuarios con el rol \textit{registrado} podrán crear contenido en el LandDebate y comentar en las entradas del blog y los debates.\\
\hline
& RLD 2.3 & Roles de usuario & El rol de usuario \textit{administrador} tendrá permisos para gestionar cualquier parte del portal y otorgar roles al resto de usuarios. Todos los usuarios con rol \textit{administrador} tendrán también el rol \textit{registrado} automáticamente. \\
\hline
& RLD 2.4 & Roles de usuario & El rol de usuario \textit{con acceso al API} será asignado por los administradores a aquellos usuarios registrados que deban tener una clave de acceso al API.  Todos los usuarios pertenecientes a éste rol también pertenecerán al rol \textit{registrado}. \\
\hline
\multicolumn{2}{|l|}{RLD 3} & Inicio de sesión & Los usuarios que previamente se hayan registrado podrán iniciar sesión en el portal utilizando su nombre de usuario y contraseña. \\
\hline
& RLD 3.1 & Inicio de sesión & El formulario de inicio de sesión será accesible desde todas las partes del portal. \\
\hline
& RLD 3.2 & Inicio de sesión & Los usuarios que se hayan registrado utilizando su cuenta de Twitter o Facebook podrán iniciar sesión utilizando un botón y no necesitarán introducir su nombre de usuario ni contraseña. \\
\hline
& RLD 3.3 & Inicio de sesión & Los usuarios registrados podrán pedir una nueva contraseña para acceder al portal en caso de haber olvidado la suya. \\
\hline
\multicolumn{2}{|l|}{RLD 4} & Funcionamiento de los eventos & El sistema permitirá la creación de eventos que tendrán lugar en una fecha determinada. \\
\hline
& RLD 4.1 & Funcionamiento de los eventos & Durante la creación de un evento será necesario introducir su título, contenido y la fecha en la que tendrá lugar. \\
\hline
& RLD 4.2 & Funcionamiento de los eventos & Durante la creación de un evento podrán seleccionarse aquellos tópicos con los que esté relacionado. \\
\hline
& RLD 4.3 & Funcionamiento de los eventos & Durante la creación de un evento podrá incluirse una imagen que acompañe al contenido. \\
\hline
& RLD 4.4 & Funcionamiento de los eventos & Los eventos podrán ser creados por cualquier usuario que cuente con el rol de \textit{registrado}. \\
\hline
& RLD 4.5 & Funcionamiento de los eventos & Los eventos podrán ser editados por su creador o por un usuario con rol de \textit{administrador}. \\
\hline
& RLD 4.5 & Funcionamiento de los eventos & Los eventos sólo podrán ser eliminados por un usuario con rol de \textit{administrador}. \\
\hline
\multicolumn{2}{|l|}{RLD 5} & Funcionamiento de las noticias & El sistema permitirá la creación de noticias.  Las noticias presentarán información que sea de interés para los miembros del portal. \\
\hline
& RLD 5.1 & Funcionamiento de las noticias & Durante la creación de una noticia será necesario introducir su título y contenido. \\
\hline
& RLD 5.2 & Funcionamiento de las noticias & Durante la creación de una noticia será posible incluir una imagen que acompañe al contenido. \\
\hline
& RLD 5.3 & Funcionamiento de las noticias & Las noticias podrán ser creadas por cualquier usuario que cuente con el rol de \textit{registrado}. \\
\hline
& RLD 5.4 & Funcionamiento de las noticias & Las noticias podrán ser editadas por su creador o por un usuario con rol de \textit{administrador}. \\
\hline
& RLD 5.5 & Funcionamiento de las noticias & Las noticias sólo podrán ser eliminadas por un usuario con rol de \textit{administrador}. \\
\hline
\multicolumn{2}{|l|}{RLD 6} & Funcionamiento de las entradas del blog & El sistema permitirá la creación de entradas en el blog.  Las entradas en el blog representan información relevante u opiniones que se emiten desde el propio Land Portal. \\
\hline
& RLD 6.1 & Funcionamiento de las entradas del blog & Las entradas del blog sólo podrán ser creadas, editadas o eliminadas por un usuario con rol de \textit{administrador}. \\
\hline
& RLD 6.2 & Funcionamiento de las entradas del blog & Durante la creación de una entrada del blog será necesario introducir su título y contenido. \\
\hline
& RLD 6.3 & Funcionamiento de las entradas del blog & Durante la creación de una entrada del blog será posible introducir una imagen que acompañe al contenido. \\
\hline
& RLD 6.4 & Funcionamiento de las entradas del blog & Durante la creación de una entrada del blog será posible seleccionar aquellos tópicos que se consideren relacionados con el contenido de la misma. \\
\hline
& RLD 6.5 & Funcionamiento de las entradas del blog & Cualquier usuario \textit{registrado} podrá incluir un nuevo comentario o replicar a un comentario ya existente en una entrada del blog. \\
\hline
\multicolumn{2}{|l|}{RLD 7} & Funcionamiento de los debates & El sistema permitirá la creación de debates.  Los debates tienen como finalidad fomentar el intercambio de ideas y la participación de los usuarios de la comunidad. \\
\hline
& RLD 7.1 & Funcionamiento de los debates & Los debates podrán ser creados por cualquier usuario \textit{registrado}. \\
\hline
& RLD 7.2 & Funcionamiento de los debates & Los debates sólo podrán ser editados por su creador o por un usuario con rol de \textit{administrador}. \\
\hline
& RLD 7.3 & Funcionamiento de los debates & Los debates sólo podrán ser eliminados por un usuario con rol de \textit{administrador}. \\
\hline
& RLD 7.4 & Funcionamiento de los debates & Los debates permitirán a un \textit{administrador} abrir o cerrar los comentarios en función de la fecha en la que el debate esté activo. \\
\hline
& RLD 7.5 & Funcionamiento de los debates & Cualquier usuario \textit{registrado} podrá crear un nuevo comentario o responder a uno ya existente en un debate, siempre que el debate esté activo. \\
\hline
& RLD 7.6 & Funcionamiento de los debates & Durante la creación de un nuevo debate será necesario incluir su título y contenido. \\
\hline
& RLD 7.7 & Funcionamiento de los debates & Durante la creación de un nuevo debate será necesario incluir las fechas entre las que el debate estará activo. \\
\hline
& RLD 7.8 & Funcionamiento de los debates & Durante la creación de un nuevo debate será posible incluir una imagen que acompañe al contenido. \\
\hline
& RLD 7.9 & Funcionamiento de los debates & Durante la creación de un nuevo debate será posible indicar los tópicos con los que está relacionado. \\
\hline
& RLD 7.10 & Funcionamiento de los debates & Durante la creación de un nuevo debate será posible indicar las regiones con las que está relacionado. \\
\hline
\multicolumn{2}{|l|}{RLD 8} & Funcionamiento de las organizaciones & El sistema permitirá la creación de organizaciones. \\
\hline
& RLD 8.1 & Funcionamiento de las organizaciones & Durante la creación de una organización será necesario incluir su nombre y una descripción larga. \\
\hline
& RLD 8.2 & Funcionamiento de las organizaciones & Durante la creación de una organización será posible incluir los tópicos con los que está relacionada. \\
\hline
& RLD 8.3 & Funcionamiento de las organizaciones & Durante la creación de una organización será posible incluir los países sobre los que trabaja. \\
\hline
& RLD 8.3 & Funcionamiento de las organizaciones & Durante la creación de una organización será posible incluir los países sobre los que trabaja. \\
\hline
& RLD 8.4 & Funcionamiento de las organizaciones & Durante la creación de una organización será posible incluir los áreas sobre los que opera. \\
\hline
& RLD 8.5 & Funcionamiento de las organizaciones & Durante la creación de una organización será necesario incluir la URL de su sitio web. \\
\hline
& RLD 8.6 & Funcionamiento de las organizaciones & Las organizaciones sólo podrán ser creadas por un usuario con rol de \textit{administrador}. \\
\hline
& RLD 8.7 & Funcionamiento de las organizaciones & Las organizaciones sólo podrán ser creadas por un usuario con rol de \textit{administrador}. \\
\hline
& RLD 8.8 & Funcionamiento de los debates & Las organizaciones sólo podrán ser creadas por un usuario con rol de \textit{administrador}. \\
\hline
\hline

 \end{longtable}
 
 
 
 
 
 
 
 
 
 
 
 
 
 
 
 
 
 
 
 
 

\subsubsection{Requisitos de la búsqueda}
En la tabla \ref{requisitos_busqueda} se muestra el listado de requisitos pertenecientes a la búsqueda.
\begin{longtable}[c]{|p{1mm}|p{14mm}|p{30mm}|p{90mm}|}
 \caption{Tabla de requisitos de la búsqueda.\label{requisitos_busqueda}}\\

 %Cabecera en la primera pagina
 \hline
 \multicolumn{4}{| c |}{Listado de requisitos de la búsqueda}\\
 \hline
 \multicolumn{2}{|c|}{Código} & Nombre & Descripción\\
 \hline
 \hline
 \endfirsthead
 
 %Cabecera en el resto de páginas
 \hline
 \multicolumn{4}{|c|}{Continuación de la tabla \ref{requisitos_busqueda}}\\
 \hline
 \multicolumn{2}{|c|}{Código} & Nombre & Descripción\\
 \hline
 \hline
 \endhead
 
 \hline
 \endfoot
 

\multicolumn{2}{|l|}{RBUS 1}  & Integración en el portal & La búsqueda deberá ser accesible desde todas las partes del portal. \\
\hline
 & RBUS 1.1 & Integración en el portal & La búsqueda contará con una vista especialmente dedicada a mostrar los resultados. \\
\hline
\multicolumn{2}{|l|}{RBUS 2}  & Integración con el LandBook & La búsqueda podrá indexar el contenido existente en el LandBook. \\
\hline
& RBUS 2.1 & Integración con el LandBook & La búsqueda podrá indexar y mostrar resultados pertenecientes a los indicadores del LandBook. \\
\hline
& RBUS 2.2 & Integración con el LandBook & La búsqueda podrá indexar y mostrar resultados pertenecientes a los países del LandBook. \\
\hline
\multicolumn{2}{|l|}{RBUS 3}  & Integración con el LandDebate & La búsqueda podrá indexar el contenido perteneciente al LandDebate. \\
\hline
& RBUS 3.1 & Integración con el LandDebate & La búsqueda podrá indexar y mostrar resultados pertenecientes a los debates del LandDebate. \\
\hline
& RBUS 3.2 & Integración con el LandDebate & La búsqueda podrá indexar y mostrar resultados pertenecientes a los eventos existentes en el LandDebate. \\
\hline
& RBUS 3.3 & Integración con el LandDebate & La búsqueda podrá indexar y mostrar resultados pertenecientes a las noticias del LandDebate. \\
\hline
& RBUS 3.4 & Integración con el LandDebate & La búsqueda podrá indexar y mostrar resultados pertenecientes a las entradas del blog. \\
\hline
& RBUS 3.5 & Integración con el LandDebate & La búsqueda podrá indexar y mostrar resultados pertenecientes a las organizaciones presentes en el LandDebate. \\
\hline
& RBUS 3.6 & Integración con el LandDebate & La búsqueda podrá indexar y mostrar resultados pertenecientes a los comentarios creados por los usuarios en los debates. \\
\hline
& RBUS 3.7 & Integración con el LandDebate & La búsqueda podrá indexar y mostrar resultados pertenecientes a los comentarios creados por los usuarios en las entradas de blog. \\
\hline
\multicolumn{2}{|l|}{RBUS 4}  & Personalización de los resultados & La búsqueda permitirá mostrar de diferente forma los resultados en función del tipo de contenido al que pertenezcan. \\
\hline
& RBUS 4.1 & Personalización de los resultados & Los resultados de la búsqueda mostrarán una etiqueta indicando de qué tipo de contenido se trata. \\
\hline
& RBUS 4.2 & Personalización de los resultados & La etiqueta de tipo de contenido presente en los resultados de la búsqueda permitirá acceder a todo el contenido del mismo tipo existente en el portal. \\
\hline
& RBUS 4.3 & Personalización de los resultados & Los resultados de búsqueda pertenecientes a un país del LandBook incluirán una imagen de la bandera de dicho país. \\
\hline
\multicolumn{2}{|l|}{RBUS 5}  & Acceso al contenido & Al pulsar sobre un resultado de la búsqueda deberá cargarse el contenido completo del mismo. \\
\hline
\multicolumn{2}{|l|}{RBUS 6}  & Priorización de resultados & La búsqueda priorizará los resultados pertenecientes al LandBook sobre los resultados pertenecientes al LandDebate. \\
\hline
\multicolumn{2}{|l|}{RBUS 7}  & Indexación de contenido & La indexación de contenido tendrá lugar de forma periódica y automática, sin necesidad de intervención humana. \\
\hline
& RBUS 7.1 & Indexación de contenidos & La indexación de contenido podrá ser ejecutada de forma manual por un usuario con rol de \textit{administrador}. \\
\hline
\hline

 \end{longtable}
 
 
 
 
 
 
 
 
 
 
 
 
 
 
 
 
 
 
 
 
 

\subsubsection{Requisitos del punto de entrada de datos}
A continuación, en la tabla \ref{requisitos_entrada_datos} se muestra el listado de requisitos pertenecientes al punto de entrada de datos al portal.
\begin{longtable}[c]{|p{1mm}|p{14mm}|p{30mm}|p{90mm}|}
 \caption{Tabla de requisitos del punto de entrada de datos.\label{requisitos_entrada_datos}}\\

 %Cabecera en la primera pagina
 \hline
 \multicolumn{4}{| c |}{Listado de requisitos de la búsqueda}\\
 \hline
 \multicolumn{2}{|c|}{Código} & Nombre & Descripción\\
 \hline
 \hline
 \endfirsthead
 
 %Cabecera en el resto de páginas
 \hline
 \multicolumn{4}{|c|}{Continuación de la tabla \ref{requisitos_entrada_datos}}\\
 \hline
 \multicolumn{2}{|c|}{Código} & Nombre & Descripción\\
 \hline
 \hline
 \endhead
 
 \hline
 \endfoot
 


\multicolumn{2}{|l|}{RPED 1}  & Almacenamiento de datos & El punto de datos guardará los datos que reciba en varios puntos de almacenamiento. \\
\hline
& RPED 1.1 & Almacenamiento de datos & El punto de entrada almacenará los datos en una base de datos relacional. \\
\hline
& RPED 1.2 & Almacenamiento de datos & El punto de entrada almacenará los datos en un servidor semántico, previa transformación de los mismos en formato RDF.   Como se ha mencionado en la sección ``\nameref{chapter02:alternativas_seleccionadas}'' perteneciente al capítulo \ref{chapter02}, el servidor semántico seleccionado ha sido Virtuoso.\\
\hline
& RPED 1.3 & Almacenamiento de datos & El punto de entrada almacenará los datos en el catálogo de datos.  Como se ha mencionado en la sección ``\nameref{chapter02:alternativas_seleccionadas}'' perteneciente al capítulo \ref{chapter02}, el catálogo de datos seleccionado ha sido CKAN. \\
\hline
& RPED 1.4 & Almacenamiento de datos & El punto de entrada podrá ser extendido con nuevos componentes de almacenamiento sin necesidad de modificar los ya existentes. \\
\hline
& RPED 1.5 & Almacenamiento de datos & Los datos que se guarden en los distintos puntos de almacenamiento serán equivalentes entre sí. \\
\hline
\multicolumn{2}{|l|}{RPED 2}  & Integridad de los datos & El punto de entrada de datos leerá la información de los nuevos catálogos de datos en un formato determinado por un XML Schema. \\
\hline
& RPED 2.1 & Integridad de los datos & La información de nuevos catálogos de datos que llegue al punto de entrada y no sea conforme al XML Schema especificado será rechazada y no se insertará en el portal. \\
\hline
& RPED 2.2 & Integridad de los datos & La inserción de datos se hará de manera transaccional, de forma que si se produce algún fallo durante el proceso no se incluya enel portal ningún dato inconsistente. \\
\hline
\multicolumn{2}{|l|}{RPED 3}  & Transformación de datos & El punto de entrada transformará la información que reciba a un modelo propio antes de ser exportada a los diferentes sistemas de almacenamiento.  INDICAR LA SECCIÓN EN LA QUE SE DETALLA EL MODELO CUANDO LO TENGA HECHO. \\
\hline
\hline

 \end{longtable}
 
 
 
 
 
 
 
 
 
 
 
 
 
 
 
 
 
 
 
 
 


\subsection{Especificación de los requisitos no funcionales}
\label{especificacion_requisitos_no_funcionales}
En la tabla \ref{requisitos_no_funcionales} se mostrará el listado de requisitos no funcionales de éste proyecto.
\begin{longtable}[c]{|p{1mm}|p{14mm}|p{30mm}|p{90mm}|}
 \caption{Tabla de requisitos no funcionales.\label{requisitos_no_funcionales}}\\

 %Cabecera en la primera pagina
 \hline
 \multicolumn{4}{| c |}{Listado de requisitos no funcionales}\\
 \hline
 \multicolumn{2}{|c|}{Código} & Nombre & Descripción\\
 \hline
 \hline
 \endfirsthead
 
 %Cabecera en el resto de páginas
 \hline
 \multicolumn{4}{|c|}{Continuación de la tabla \ref{requisitos_no_funcionales}}\\
 \hline
 \multicolumn{2}{|c|}{Código} & Nombre & Descripción\\
 \hline
 \hline
 \endhead
 
 \hline
 \endfoot
 

\multicolumn{2}{|l|}{RNF 1} & Interfaz del sistema & La interfaz del sistema contará con una apariencia \textit{flat} conforme a las últimas tendencias de diseño web. \\
\hline
 & RNF 1.1 & Interfaz del sistema & La interfaz del sistema contará con un diseño \textit{responsive} que permita su adaptación a distintos tamaños de pantalla. \\
\hline
 & RNF 1.2 & Interfaz del sistema & La interfaz del sistema contará con un diseño unificado a lo largo de todas las secciones que lo componen. \\
\hline
\multicolumn{2}{|l|}{RNF 2} & Seguridad del punto de entrada de datos & El punto de entrada de datos permitirá utilizar una lista blanca con la que restringir los orígenes de las peticiones de entrada de datos. \\
\hline
\multicolumn{2}{|l|}{RNF 3} & Escalabilidad del punto de entrada de datos & El punto de entrada de datos deberá ser escalable para permitir la inserción de grandes volúmenes de datos. \\
\hline
 & RNF 3.1 & Escalabilidad del punto de entrada de datos & El punto de entrada de datos podrá procesar catálogos de datos con hasta 500.000 observaciones. \\
\hline
\multicolumn{2}{|l|}{RNF 4} & Rendimiento del LandBook & El LandBook permitirá cachear las consultas que se realicen a la base de datos con el objetivo de maximizar el rendimiento de las visualizaciones de datos. \\
\hline
\hline

 \end{longtable}
 
 
 
 
 
 
 
 
 
 
 
 
 
 
 
 
 
 
 
 
 