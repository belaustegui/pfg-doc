En esta sección se identificarán todos los actores del sistema.  Se consideran actores todas aquellas personas, organizaciones o elementos que toman algún papel en el sistema.

Existirán seis tipos de actores que interactúan con el sistema, dos de estos actores no serán personas, si no otros componentes de software.
\begin{description}
\item[Usuarios anónimos]  Los usuarios anónimos son todos aquellos usuarios del portal que no hayan iniciado sesión con una cuenta de usuario.  El papel de estos usuarios en el sistema será el de consumidores de información, puesto que únicamente se les permitirá visualizar los contenidos de la zona de datos y de la zona social (exceptuando acceder a la información de los perfiles de otros usuarios registrados).  Estos usuarios también podrán utilizar la búsqueda, registrar una nueva cuenta de usuario o iniciar sesión con una cuenta de usuario ya existente.
\item[Usuarios registrados]  Los usuarios registrados son todos aquellos usuarios que tienen una cuenta de usuario en el portal y que además han iniciado sesión con ella.  El papel de estos usuarios será tanto de consumidores como creadores de información, puesto que además de ver los contenidos de la zona de datos y la zona social (incluyendo acceder a la información de los perfiles de otros usuarios registrados) también podrán aportar nueva información a la zona social.  Concretamente podrán crear debates, eventos y noticias, además de comentar en los debates abiertos o en las entradas del blog.
\item[Usuarios con acceso al API]  Los usuarios con acceso al API son usuarios registrados que además cuentan con una clave de acceso al API pública del portal.  Estos usuarios jugarán un papel de creadores, consumidores y difusores de información, puesto que además de todas las capacidades de los usuarios registrados también tendrán un acceso total al API que podrán utilizar para crear servicios o aplicaciones externas que se beneficien de los datos ofrecidos por el nuevo Land Portal.
\item[Administradores]  Los administradores serán aquellos usuarios de confianza que se encarguen de mantener el funcionamiento del nuevo Land Portal.  Estos usuarios tendrán principalmente un papel de moderadores de la zona social del portal.  Tendrán capacidad para editar o eliminar los debates, noticias o eventos creados por otros usuarios; abrir o cerrar los debates para que el resto de usuarios puedan participar en ellos; moderar o eliminar los comentarios introducidos por otros usuarios; publicar o modificar contenido en el blog de Land Portal; gestionar el contenido del catálogo de datos y otorgar o eliminar las capacidades de administración o acceso al API del resto de usuarios registrados.
\item[Importadores de datos]  A diferencia de los actores descritos anteriormente, los importadores de datos no serán personas, si no que serán aplicaciones creadas con el fin de insertar nuevos datos en el portal.  El objetivo de estas herramientas será capturar datos provenientes de diferentes fuentes u organizaciones, transformar dichos datos a un XML Schema definido y enviarlos al punto de entrada de datos del portal.
\item[Visualizaciones de datos]  De la misma forma que sucede con los importadores de datos, las visualizaciones serán aplicaciones creadas con el fin de extraer datos del portal.  El objetivo de las visualizaciones será transformar los datos contenidos en el portal en representaciones visuales que resulten atractivas para los usuarios.
\end{description}


