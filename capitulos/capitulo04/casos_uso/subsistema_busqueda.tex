En esta sección se detallarán los casos de uso y escenarios pertenecientes al subsistema de búsqueda. La figura \ref{fig:casos_uso_subsistema_busqueda} muestra el diagrama de casos de uso de dicho subsistema.

\begin{figure}[h]
\centering
\includegraphics[width=\textwidth]{casos_uso/diagrama_casos_uso_busqueda}
\caption{Diagrama de casos de uso del subsistema de búsqueda}
\label{fig:casos_uso_subsistema_busqueda}
\end{figure}

\subsubsection{Caso de uso ``realizar una búsqueda''}
\begin{description}
\item[Descripción] Un usuario busca información en el sistema.
\item[Actores] Cualquier rol de usuario registrado o no en el sistema.
\item[Escenario principal] \hfill
							\begin{enumerate}
							\item El usuario introduce un texto en el formulario de búsqueda
							\item El usuario pulsa el botón de buscar
							\item El sistema devuelve los resultados correspondientes
							\end{enumerate}						
\end{description}

\subsubsection{Caso de uso ``actualizar índice de contenidos''}
\begin{description}
\item[Descripción] El administrador actualiza el índice de contenidos del subsistema de búsqueda.
\item[Actores] El administrador del sistema.
\item[Escenario principal] \hfill
							\begin{enumerate}
							\item El administrador accede al panel de administración.
							\item Una vez en el panel de administración, accede a las opciones de la búsqueda
							\item El administrador pulsa el botón correspondiente para realizar la regeneración del índice de contenidos.
							\item El sistema actualiza el índice de contenidos incluyendo los nuevos contenidos y eliminado los contenidos que ya no existan.
							\end{enumerate}						
\end{description}