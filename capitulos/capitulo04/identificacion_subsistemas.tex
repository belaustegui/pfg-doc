Una vez identificados los interesados tanto directos como indirectos y los requisitos del sistema se procederá a realizar una descomposición en subsistemas.  Cada subsistema tendrá una funcionalidad única y acotada.\newline
A continuación se detalla cada uno de los subsistemas identificados:

\begin{description}
\item[Subsistema de búsqueda]  El subsistema de búsqueda es el encargado de gestionar todas las búsquedas que realicen los usuarios y retornar los resultados convenientes.  Además también se encargará de indexar periódicamente (o puntualmente si un administrador lo desea) los contenidos del portal.
\item[Subsistema de gestión de usuarios]  Éste subsistema será el encargado de gestionar todas las operaciones que se realizan relacionadas con los datos de un usuario.  Algunas de las tareas gestionadas por este subsistema son los registros, los inicios de sesión o la asignación de diferentes roles de usuario.
\item[Subsistema de gestión de debates]  Éste subsistema se encargará de gestionar todas las operaciones relacionadas con los debates, desde la creación y modificación de nuevos debates hasta la apertura o cierre de los debates ya existentes y la gestión de los comentarios de los usuarios.
\item[Subsistema de gestión de eventos]  Éste subsistema se encargará de gestionar todas las operaciones que guarden relación con los eventos.  Las tareas típicas de éste subsistema serán la creación y la edición de eventos.
\item[Subsistema de gestión de noticias]  Éste subsistema se encargará de gestionar las operaciones relacionadas con las noticias, principalmente la creación y edición de noticias.
\item[Subsistema de gestión del blog]  Éste subsistema se encargará de gestionar las tareas relacionadas con las entradas del blog.  Algunas funciones de éste subsistema serán la creación y edición de entradas en el blog y la moderación de los comentarios de usuarios.
\item[Subsistema de gestión de datos]  Éste subsistema será el encargado de gestionar los datos de la parte de datos del portal.  Las funciones más representativas de éste subsistema son la inserción de nuevos catálogos de datos y la provisión de información con la que crear las visualizaciones.
\end{description}


\subsection{Descripción de las interfaces entre subsistemas}