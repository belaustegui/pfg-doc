\begin{longtable}[c]{|p{1mm}|p{14mm}|p{30mm}|p{90mm}|}
 \caption{Tabla de requisitos no funcionales.\label{requisitos_no_funcionales}}\\

 %Cabecera en la primera pagina
 \hline
 \multicolumn{4}{| c |}{Listado de requisitos no funcionales}\\
 \hline
 \multicolumn{2}{|c|}{Código} & Nombre & Descripción\\
 \hline
 \hline
 \endfirsthead
 
 %Cabecera en el resto de páginas
 \hline
 \multicolumn{4}{|c|}{Continuación de la tabla \ref{requisitos_no_funcionales}}\\
 \hline
 \multicolumn{2}{|c|}{Código} & Nombre & Descripción\\
 \hline
 \hline
 \endhead
 
 \hline
 \endfoot
 

\multicolumn{2}{|l|}{RNF 1} & Interfaz del sistema & La interfaz del sistema contará con una apariencia \textit{flat} conforme a las últimas tendencias de diseño web. \\
\hline
 & RNF 1.1 & Interfaz del sistema & La interfaz del sistema contará con un diseño \textit{responsive} que permita su adaptación a distintos tamaños de pantalla. \\
\hline
 & RNF 1.2 & Interfaz del sistema & La interfaz del sistema contará con un diseño unificado a lo largo de todas las secciones que lo componen. \\
\hline
\multicolumn{2}{|l|}{RNF 2} & Seguridad del punto de entrada de datos & El punto de entrada de datos permitirá utilizar una lista blanca con la que restringir los orígenes de las peticiones de entrada de datos. \\
\hline
\multicolumn{2}{|l|}{RNF 3} & Escalabilidad del punto de entrada de datos & El punto de entrada de datos deberá ser escalable para permitir la inserción de grandes volúmenes de datos. \\
\hline
 & RNF 3.1 & Escalabilidad del punto de entrada de datos & El punto de entrada de datos podrá procesar catálogos de datos con hasta 500.000 observaciones. \\
\hline
\multicolumn{2}{|l|}{RNF 4} & Rendimiento del LandBook & El LandBook permitirá cachear las consultas que se realicen a la base de datos con el objetivo de maximizar el rendimiento de las visualizaciones de datos. \\
\hline
\hline

 \end{longtable}
 
 
 
 
 
 
 
 
 
 
 
 
 
 
 
 
 
 
 
 
 